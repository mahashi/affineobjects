\begin{lemma}\label[lemma]{7cc}
Let $Y = \rsite{C}{T}{O}$ be a ringed site.
The functor $\pstilde$ is left adjoint to 
\[\globalsections:\pshmodules{\sheaf{O}} \rightarrow \modules{R}.\]
\end{lemma}

\begin{proof}
Let $a$ be an object of $C$.
Let $M,N$ be $R$-modules. 
Let $\presheaf{F},\presheaf{G} \in \pshmodules{\sheaf{O}}$ be presheaf modules.

Let $\varphi:\pstilde{M} \rightarrow \presheaf{G}$ be a morphism of presheaf modules.
Let $\phi: M \rightarrow \globalsections{G}$ be a morphism of presheaf modules.

Define
\begin{align*}
\alpha = H_{M,\presheaf{G}} :& 
	\Hom{\pstilde{M}}{\presheaf{G}} \rightarrow \Hom{M}{\globalsections{G}} \\
	:& \varphi \mapsto \varphi_1
\end{align*}

where $\varphi_1$ is the component of $\varphi$ on the global sections.

Define 
\[\beta = L_{M,\presheaf{G}}: 
	\Hom{M}{\globalsections{G}} \rightarrow \Hom{\pstilde{M}}{\presheaf{G}}
\] 
by
\[ \beta(\phi)_a = \phi \tensor_R \sections{a}{O}.\]

We will show that $\beta$ and $\alpha$ are mutually inverse. 

Let $d = \beta(\alpha(\varphi))$. 
Let $m \tensor g \in M \tensor_R \sections{a}{O}$.
Let $p: \pstilde(M)(1)\rightarrow \pstilde(M)(a)$ be the projection map.
Let $q: \presheaf{G}(1)\rightarrow \presheaf{G}(a)$ be the projection map.
Then $d_a(m \tensor g) = \varphi_1(m) \tensor g$
and

\begin{align*}
	\varphi_a(m \tensor g) &=  g \varphi_a(m \tensor 1) \reason{linearity}\\
		&= g \varphi_a(p(m)) \\
		&= g q(\varphi_1(m)) \reason{naturality of $\varphi$}\\
		&= g (\varphi_1(m) \tensor 1) \\
		&= \varphi_1(m) \tensor g.
\end{align*}

Hence $d = \varphi$. 
In words, the natural transformations from presheaves of the from $\pstilde(M)$ 
are uniquely determined by their global sections component.

Let $d = \alpha(\beta(\phi))$. 
Let $m \in M$.
Then $d(m) = (\phi \tensor_R R)(m) = \phi(m)$.
Hence $d = \phi$, which makes $H$ and $L$ mutual inverses.

Now we will show naturality in $M$ and $\presheaf{G}$.
Let $g:N\rightarrow M$ and $h:\presheaf{F} \rightarrow \presheaf{G}$.
Let $\rho \in \Hom{\pstilde(N)}{\presheaf{F}}$.
Let $k = H_{M,\presheaf{G}} (h \circ \rho \circ \pstilde(f))$.
Let $l = h_1 \circ H_{N,\presheaf{F}}(\rho) \circ f$.

Unfolding the definition for $H$ shows that $k = h_1\rho_1 f$
and $l = h_1\rho_1 f$ as well.
This proves naturality in $M$ and $\presheaf{G}$ 
and the adjunction between $\pstilde$ and $\globalsections$
between $\modules{R}$ and $\pshmodules{\sheaf{O}}$.
\end{proof}
