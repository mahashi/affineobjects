\begin{lemma}\label[lemma]{81e}
Let $f:\rsite{D}{S}{U} \rightarrow \rsite{C}{T}{O}$ be a morphism of ringed site.
Let $f^#$ be an isomorphism.
We get a natural isomorphism 
$t:\direct{f} \circ \pstilde \Rightarrow \pstilde \circ (\blank \tensor_{\sect{\terminal}{O}} \sections{\terminal}{U})$.
\end{lemma}

\begin{proof}
Define the natural transformation
$t: \pstilde \circ (\blank \tensor_{\sect{1}{O}} \sections{1}{U}) \Rightarrow \direct{u} \circ \pstilde,$
by 
for each $\sect{1}{O}$-module $M$
and for each $a \in \cat{D}$,
\[t_{M,a}: M \tensor_{\sect{1}{O}} \sect{1}{U} \tensor_{\sect{1}{U}} \sect{a}{U}  
	\rightarrow  M \tensor_{\sect{1}{O}} \sect{a}{O} ,\]
\[m\tensor r \tensor s \mapsto m \tensor rf^{\#,-1}(s).\]

Every component $t_{M,f}$ is an isomorphism by basic commutative algebra and the fact that $f^{\#}$ is an isomorphism
\end{proof}

\begin{example}
The morphism of ringed sites $u$ is an example where \Cref{81e} holds. 
\end{example}

\begin{lemma}
The functor $\direct{f}$ is left adjoint to $\inverseimg{f}$.
\end{lemma}
\begin{proof}
%TODO: insert reference
\end{proof}

\begin{corollary}
The functor $\direct{f}$ commutes with arbitrary colimits.
\end{corollary}

\begin{lemma}
Let $f^#$ be an isomorphism.
Let $\sheaf{F}$ be a quasi-coherent sheaf of modules.
Then $\direct{f}\sheaf{F}$ is quasi-coherent.
\end{lemma}
\begin{proof}
%TODO: insert reference to left adjointness of \direct{f} and finish proof
\end{proof}

\begin{lemma}
Let $f:\rsite{D}{S}{U} \rightarrow \rsite{C}{T}{O}$ be a morphism of ringed site
that is cover lifting.
Let $f^#$ be an isomorphism.
Let $\sheaf{F}$ be a quasi-coherent module on $\cat{C}.$
Let $M = \sect{1}{F}$.
Consider $\counit:\stilde{M} \rightarrow \sheaf{F}$.
Then $\direct{f}\counit: \stilde{M\tensor_{\sect{\terminal}{O}} \sections{\terminal}{U})} \rightarrow \direct{f}\sheaf{F}$ is 
the counit of the adjunction $\adj{\stilde}{\sect{1}}$ on $\cat{D}$.
\end{lemma}

\begin{proof}
Let $a\in\cat{D}$.
We will show that $\direct{f}\counit$ corresponds to the same
morphism $\pstilde{M\tensor_{\sect{\terminal}{O}}} \rightarrow \direct{f}\sheaf{F}$
as the counit. By the universal property of  the sheafification, this implies that $\direct{f}\counit$ is the counit.

We have
$(\direct{f}\counit)_a (\omega^2(m\tensor r)) = rm$, which is the same as for the counit.
\end{proof}