\begin{lemma}\label[lemma]{842}
Let $R$ be a ring.
Let $M$ be a $R$-module.
Consider $\pstilde{M},\stilde{M}$ as sheaf modules on $\opens{\Spec{R}}$.
Then $$\omega^2_{\pstilde{M},\Spec{R}}: \Gamma(\Spec{R};\pstilde{M}) \rightarrow \Gamma(\Spec{R};\stilde{M})$$ is an isomorphism.
\end{lemma}

\begin{proof}
Let $\Ip \subset R$ be a prime ideal.
As stated in \Cref{6f4}, we may use results from \cite[Tag 01BH]{stacks} in this setting.
We will use that $\stilde{M}_{\Ip} = M \tensor_R R_{\Ip}$.
By naturality, localized at $\Ip$,
the map $\omega^2$ sends $m$ to $m\tensor 1 \in M \tensor_R R_{\Ip}$,
hence is the inverse of the multiplication map which is an isomorphism.
Hence globally $\omega^2$ is an isomorphism.
\end{proof}

\begin{corollary}\label[corollary]{853}
Let $X$ be a scheme.
Consider $\pstilde{M},\stilde{M}$ as sheaf modules on $\opens{X}$.
Let $\Spec{R}$ be an open subset of $X$.
Let $M$ be a $\sect{X}{O}$-module.
Then $\omega^2_{\pstilde{M},\Spec{R}}: \Gamma(\Spec{R};\pstilde{M}) \rightarrow \Gamma(\Spec{R};\stilde{M})$ is an isomorphism. 
\end{corollary}


Note that $\unit_M$ is equal to $\omega^2_{\pstilde{M},\Spec{R}}$ so we get the following.

\begin{corollary}\label[corollary]{85e}
Consider the adjunction $\mainadjunction$ on $\opens{\Spec{R}}$.
The unit $\unit_M:M \rightarrow \Gamma(\Spec{R};\stilde{M})$ is an isomorphism.
\end{corollary}