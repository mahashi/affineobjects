\begin{definition}[k]\label[definition]{a5b}
Let $X$ be a scheme.
Define the functor $k:\opens{X} \rightarrow \overschemes{X}$
by $U \mapsto ((U,\sheaf{O}_U),i)$
where $i: U\rightarrow X$ is the inclusion of the open subscheme into $X$.

We will show that it preserves limits and covers.
The terminal $X \in \opens{X}$ is sent to the terminal $X \rightarrow X$.
Let $U\rightarrow V$ and $W\rightarrow V$ be two morphism in $\opens{X}$.
We have $k(U\intersect W) = U \intersect W \rightarrow X$ which is the pullback
of $k(U) \rightarrow k(V)$ and $k(W) \rightarrow k(V)$. 

Let $S= \set{D(f_i) \rightarrow \Spec{R}}$ be one of the generating family in $\opens{X}$.
Note that $k(D(f_i))$ is isomorphic to the object $\Spec{R_{f_i}} \rightarrow \Spec{R}$.
Hence $k(S)$ generates a covering sieve on $\schemes$ and hence on $\overschemes{X}$.
We established that $k$ is a morphism of sites.

Consider $\rsite{\opens{X}}{T}{O}$ and $\rsite{\overschemes{X}}{S}{U}$.
Note that $\direct{k}\sheaf{U} = \sheaf{O}$ by construction.
Define the morphism of ringed sites $(k,\id): \rsite{\opens{X}}{T}{O} \rightarrow \rsite{\overschemes{X}}{S}{U}$.
We will denote this by $k$.
\end{definition}