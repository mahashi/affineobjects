\begin{lemma}\label[lemma]{4ae}
For any $X \in \LocRingSpaces$ and $R\in \Rings$
we have an isomorphism
\[\Hom{X}{\Spec{R}}{\LocRingSpaces} \rightarrow \Hom{R}{\sect{X}{O}}{\Rings}\]
that is natural in $X$ and $R$.
In short \[\Spec: \Rings \rightarrow \LocRingSpaces\]
is adjoint to \[\globalsections: \LocRingSpaces \rightarrow \Rings.\]
\end{lemma}

\begin{proof}
Sending a morphism of locally ringed spaces $X \rightarrow \Spec R$ to its global component $R \rightarrow \sect{X}{O}$
will turn out to be an isomorphism.
The inverse is the following map. 

Let $\varphi: R \rightarrow \sect{X}{O}$ be given.
We need to construct a morphism of locally ringed spaces $(f,f^{\#})$.
Define $f(x) = \ker(x)$.
For distinguised open $D(f)\subset \Spec{R}$,
define $f^{\#}_{D(f)}(\frac{s}{f}) = \frac{s}{f}$.
This makes sense because $f \in \sect{X}{O}$ is invertible in $D_X(f)$

These maps are mutually inverses and we have naturality.
See \cite[Tag 01I1]{stacks}
\end{proof}