%%When and why can we use covering families instead of sieves?
\subsection{Topology}

\begin{definition}[Sieve]
A sieve on $X\in \cat{C}$ is a subpresheaf(or subobject or subfunctor) of the representable presheaf $\ynd{X}$. 
The maximal sieve will be denoted $\max(C)$.
\end{definition}


\begin{definition}[Grothendieck Topology]
A Grothendieck topology $\topo{T}$ is a family $\topo{T}(X)$ of 'covering' sieves for every $X\in \cat{C}$ with the following conditions:
\begin{itemize}
\item  $\max(X) \in \topo{T}(X)$
\item $f^*R\in \topo{T}(X')$ if $R\in \topo{T}(X)$ for any $f:X'\rightarrow X$
\item if $f^*R\in \topo{T}(X')$ for all $f\in S$ with $S\in \topo{T}(X)$ then $R\in \topo{T}(X)$
\end{itemize}
Note that if $f\in R$ then $f^*R=\max(X')$. So if $R\subset S$ and $R$ is covering then $S$ is covering. 
Also $R\intersect S$ is covering \iff $R$ and $S$ are covering.
\end{definition}

\begin{definition}[Basis]
Let $\cat{C}$ have pullbacks.
A Grothendieck pretopology $\basis{B}$ is a collection $\basis{B}(X)$ of families $\{f_i:X_i\rightarrow X\}$ of 'covering' morphisms for every $X\in \cat{C}$ with the following conditions:
\begin{itemize}
\item  every isomorphism is a covering singleton family.
\item (Stability) The pullback of a covering family is covering. 
If  $\{f_i:X_i\rightarrow X\}$ is covering and $g:Y\rightarrow X$, then $\{f'_i:X_i\times_X Y\rightarrow Y\}$ is covering.
\item (Transitivity) If $\{f_i:X_i\rightarrow X\}$ is covering and $\{f_{ij}:X_{ij}\rightarrow X_i\}$ for every $i$, then $\{f_{ij}:X_{ij}\rightarrow X\}$ is covering.
\end{itemize}

Generating a real topology: take any sieve containing a covering family to be a covering sieve.
Any sieve is generated by itself as covering family, in this way any topology can be interpreted as a pretopology. This enables one to use the pullbacks in proofs.
\end{definition}

\begin{definition}
A site $\site{C}{T}$ is a category $\cat{C}$ with the Grothendieck topology $\topo{T}$.
If $\cat{C}$ has pullbacks, then we consider $\topo{T}$ always as a pretopology.
\end{definition}

\begin{definition}[Cocontinuous functor]


\end{definition}

\begin{lemma}
Let $\site{C}{T}\xrightarrow{g} \site{D}{S}$. Let $\presheaf{F}$ be a presheaf on $\cat{D}$.
If $g$ is cocontinuous, then 

\[\forward{g}{\plus{F}} \iso \plus{\forward{g}{F}}.\]
\end{lemma}
\begin{proof}
Let $X$ be an object.
The two presheafs reduce to
\[ \limit{R\in S(g(X))} \Hom{R}{F} \rightarrow \limit{K\in T(X)} \Hom{g(K)}{F}.\]

The poset of covering sieves on $X$ is send to a dense
poset of $g(X)$ so the limits are isomorphic and this isomorphism is natural.
\end{proof}