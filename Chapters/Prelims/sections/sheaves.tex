%References: Moerdijk, Shapira
%todo: sheaves with values in arbitraty categories.
\subsection{Sheaves}

\begin{definition}[Sheaves of sets]
Let $(\cat{C},\topo{T})$ be a site. Let $\presheaf{F}\in \presheaves{C}$.
A compatible family on $X$ is a family of elements $x_f\in \presheaf{F}(X_f)$ indexed by a sieve $R$ on $X$, where $X_f=\domain{f}$ 
and such that $\pullback{g}(x_{f})=x_{fg}$. This is the same as a morphism $R\rightarrow \presheaf{F}$ as presheaves.

An amalgamation of a compatible family $(x_f)_R$ on $X$ is an element $x\in \presheaf{F}(X)$ such that $\pullback{f}(x)=x_f$.
Hence given an morphism $X\rightarrow \presheaf{F}$ that extends the morphism $R\rightarrow \presheaf{F}$ defined by the compatible family.

%%TODO: pullback sequence

A presheaf that admits a unique amalgamation of every compatible family is called a sheaf.
The category $\sheaves{C}$ is the full subcategory on these sheaves. 
Let $i$ be the inclusion functor $\sheaves{C}\rightarrow \presheaves{C}$.
\end{definition}

\begin{definition}[Sheaves \#2]
Let $(\cat{C},\topo{T})$ be a site. Let $\presheaf{F}\in \presheaves{C}{A}$. 
Let $\cat{A}$ be (small) complete. Define $\presheaf{F}(R)$ for a sieve $R$ on $X$ to be 
\[\Hom{R}{F}.\]

We call $\presheaf{F}$ a sheaf if the map 
\[\presheaf{F}(X) \rightarrow \presheaf{F}(R)\] is an isomorphism.

This is just shorthand notation for the above definition. There is a bijection between $\Hom{R}{F}$ and matching families and the map sends a section to the unique matching family indexed by $R$ it is an amalgamation of.
\end{definition}

\begin{definition}[Plus construction Shapiro]
Let $(\cat{C},\topo{T})$ be a site. Let $\presheaf{F}\in \presheaves{C}$.
Define the category $\cat{I}$ whose objects are pairs $(R,\varphi)$ with $R\in \topo{T}(X)$ and $R\xrightarrow{\varphi} F$.
A morphism between $(R,\varphi)\rightarrow (S,\phi)$ are inclusions $R\rightarrow S$ such that $\varphi=\phi$ restricted to $R$.

Then
\[\plus{F}(X) = \limit{(R,\varphi)\in \cat{I}} F(R).\]

More concretely, $\plus{F}(X)$ is the set of all objects $(R,\varphi)$ with the equivalence relation that $(R, \varphi)\sim (S,\phi)$ if $\varphi=\phi$ on $R\cap S$. Or equivalently the set all compatible families with the equivalence relation that $(x_f)_R \sim (y_g)_S$ if $(x_f)_{R\cap S} = (y_g)_{R\cap S}$. 

We have the map
\[\unit:\presheaf{F}(X)\rightarrow \product{f\in R} \presheaf{F}(\domain{f})\]
\[x \mapsto (X,x).\]

This defines a natural transformation from $\id$ to $\plus{\blank}$.
\end{definition}

\begin{definition}[Plus construction Moerdijk]
%TODO: fill in moerdijks construction
\end{definition}

\begin{lemma}
$\plus{F}$ is separated
\end{lemma}

\begin{lemma}
If $\presheaf{F}$ is separated then $\plus{F}$ is a sheaf.
\end{lemma}


\begin{lemma} 
Let $\site{C}{S} \xrightarrow{g} \site{D}{T}$. 
Let $\presheaf{F}$ be a presheaf on $\cat{D}$. 
Then 
\[ \plus{F}g \iso \plus{Fg}.\]
\end{lemma}

\begin{proof}

\end{proof}

\begin{definition}[Sheafification]
Define $\sheafify{F}=\plus{\plus{F}}: \presheaves{C} \rightarrow \sheaves{C}$. 
This is a left adjoint to the inclusion functor.

The functor $\sheafify{\blank}$ commutes with finite limits. There are two different proofs: Moerdijk and stack+Shapiro. Shapiro defines the plus with as a colimit over a directed set, hence this commutes with limits.
\end{definition}

\begin{theorem}
The following this are equivalent for a category $\cat{C}$.
\begin{itemize}
\item A Grothendieck topology
\item A full subcategory $\cat{E}\subset \presheaves{C}$ such that the the inclusion functor has a left adjoint that preserves finite limits.
\end{itemize}
\end{theorem}

%% Properties
\begin{remark}[Properties]
\begin{itemize}
\item The restriction of a sheaf is a sheaf
\end{itemize}
\end{remark}

%% Rings

%% Modules

%% Quasi-coherent modules