\begin{remark}
The next result will prove that our definition of quasi-coherence in the sense of \Cref{77a}
coincides with the usual definition for schemes using $\widetilde{\blank}$.
See \cite[definition 13.2.2]{vakil} for the usual definition.
\end{remark}

\begin{lemma}\label[lemma]{tildeqc}
Let $\sheaf{F}$ be a sheaf of modules on scheme $X$.
$\sheaf{F}$ is quasi-coherent on $X$ 
 \iff 
for any open $\Spec{R} \subset X$ 
the sheaf $\restr{\Spec{R}}{\sheaf{F}}$ is isomorphic to $\widetilde{\sect{\Spec{R}}{F}}$.
\end{lemma}

\begin{proof}
$\Rightarrow$:
By assumption we get local presentations indexed by a covering.
Let $\Union_{i\in I} U_i = X$ be this covering.
Assume without loss of generality that it is an affine open covering.
Let $U_i = \Spec{R_i}$.
Let $\sheaf{O}_{U_i}^K \rightarrow \sheaf{O}_{U_i}^J \rightarrow \restr{U_i}{\sheaf{F}} \rightarrow 0$
be one of the given presentations.
Taking global sections gives us an exact sequence
\[R_i^K \rightarrow R_i^J \rightarrow \sect{U_i}{F} \rightarrow 0.\]
Tensoring it with the localisation $R_{i,f}$ for any $f\in R_i$ yields
\[R_{i,f}^K \rightarrow R_{i,f}^J \rightarrow \sect{U_i}{F}\tensor R_{i,f} \rightarrow 0.\]
Taking sections at $D(f)$ from the sheaf sequence yields
\[R_{i,f}^K \rightarrow R_{i,f}^J \rightarrow \sect{D(f)}{F} \rightarrow 0.\]
Hence $\restr{U_i}{\sheaf{F}}$ is the unique sheaf with 
$D(f) \mapsto \sect{U_i}{F}_f$,
which we defined to be $\widetilde{\sect{U_i}{F}}$.
By the affine communication lemma, this property holds for any affine and not just for the affines in this covering.

$\Leftarrow$:
Take a presentation $R^I \rightarrow R^J \rightarrow M$
and apply $\widetilde{\blank}$. 
Then note that $\widetilde{\blank}$ commutes with infinite coproducts
and $\widetilde{R} = \sheaf{O}_{\Spec{R}}$.
So we get a presentation 
$\sheaf{O}_{\Spec{R}}^I \rightarrow \sheaf{O}_{\Spec{R}}^J \rightarrow \sheaf{F}$ 
on every affine open subset $\Spec{R}\subset X$, hence $\sheaf{F}$ is quasi-coherent.
\end{proof}