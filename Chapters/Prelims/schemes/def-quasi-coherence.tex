The next result will prove that our definition of quasi-coherence in the sense of \Cref{77a}
coincides with the usual definition for schemes using $\widetilde{\blank}$.
See \cite[13.2.2]{vakil} for the usual definition.

\begin{lemma}\label[lemma]{tildeqc}
Let $\sheaf{F}$ be a sheaf of modules on scheme $X$.
Then $\sheaf{F}$ is quasi-coherent on $X$ 
 \iff 
for any open $\Spec{R} \subset X$ 
the sheaf $\restr{\Spec{R}}{\sheaf{F}}$ is isomorphic to $\widetilde{M}$ 
for some $R$-module $M$.
\end{lemma}

\begin{proof}
$\Rightarrow$:
%TODO: proof

$\Leftarrow$:
Take a presentation $R^I \rightarrow R^J \rightarrow M$
and apply $\widetilde{\blank}$. 
Then note that $\widetilde{\blank}$ commutes with infinite coproducts
and $\widetilde{R} = \sheaf{O}_{\Spec{R}}$.
So we get a presentation 
$\sheaf{O}_{\Spec{R}}^I \rightarrow \sheaf{O}_{\Spec{R}}^J \rightarrow \sheaf{F}$ 
on every affine open subset $\Spec{R}\subset X$, hence $\sheaf{F}$ is quasi-coherent.
\end{proof}