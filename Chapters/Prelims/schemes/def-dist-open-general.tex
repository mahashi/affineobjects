\begin{definition}[General distinguised open]\label[definition]{87f}
Let $X$ be a scheme. We will generalize distinguised opens to general schemes.
Let $f\in \sect{X}{O}$.
Define $D_X(f)$, or what \cite{harts} calls $X_f$, for $X$ to be the open subscheme
defined by $x\in D_X(f)$ \iff $f\not\in \ker{x}$.

Let $X = \Union_i U_i$ be a affine covering of $X$ with $U_i = \Spec{R_i}$.
Equivalently we can define $D_X(f) = \Union_i D_{X,i}(f)$, 
where $D_{X,i}(f)$ is the distinguised open of $U_i$ 
that corresponds to the image of $f$ in $R_i$.

When no confusion can arise, we leave the subscript $X$ out.
\end{definition}

\begin{lemma}\label[lemma]{88d}
Let $f\in \sect{X}{O}$.
The restriction of $f$ is invertible on $D(f)$.
\end{lemma}
\begin{proof}
We know that $f$ is invertible in the stalk of each point of $D(f)$.
This gives us a compatible family of local inverses.
By the sheaf axiom these have a global amalgamation, which is the inverse of $f$.
\end{proof}