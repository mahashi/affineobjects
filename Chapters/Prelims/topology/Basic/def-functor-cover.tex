\begin{definition}\label[definition]{9be}
Let $G:\cat{C} \rightarrow \cat{D}$ be a functor between sites.
Let $c\in \cat{C}$ and let $R$ be a covering sieve on $c$.
$G$ induces a map $\sieves{c} \rightarrow \sieves{G(c)}$
by sending a sieve $R$ to the sieve generated by $G(R)$.
The functor $G$ is said to preserves covers or called cover-preserving
if the induced map restricts to a map $\cov{c} \rightarrow \cov{G(c)}$.
See \cite[C2.3]{elephant}.

The functor $G$ is said to lift cover or have the covering lifting property
if for every $R\in \cov{G(c)}$ there is some $S\in \cov{c}$ such that $G(S)\subset R$.
See \cite[420]{sheavGeomLogic}.
\end{definition}
