% \begin{definition}[Sieve]
% Let $\cat{C}$ be a category 
% and $a \in \cat{C}$.
% A sieve $S$ on $a$ is a subpresheaf of $\y{a}$.
% Explicitly, for each $c\in \cat{C}$, 
% $\sieve{S}{c}$ is a subset of $\Hom{c}{a}$
% such that $fg \in \sieve{S}(\dom{g})$ 
% for all $f \in \sieve{S}{c}$
% and for all $g \in \y{c}$.

% The maximal sieve on $a$, 
% which is $\y{a}$, 
% will be denoted by $\maxsieve{a}$.
% \end{definition}

\begin{definition}[Sieve]
Let $\cat{C}$ be a category 
and $a \in \cat{C}$.
Define the maximal sieve $\maxsieve{a}$ on $a$ to be the set of all morphisms to $a$.
In formula,
\[\maxsieve{a} = \set{ f \in \cat{C} \suchthat \codomain{f} = a}.\]

A sieve $S$ is a subset of $\maxsieve{a}$ such that 
$gf \in S$ for any $f\in S$ and any $g$.
\end{definition}

\begin{remark}
Let $\cat{C}$ be a category 
and $a,b \in \cat{C}$.
Let $f:b\rightarrow a \in \overcat{C}{a}$.

Any morphism to $b$ is also a morphism to $f$ and vice versa.
This observation yields us that $\sievecat{b} = \sievecat{f}$.
Moreover composition in $\cat{C}$ and $\overcat{C}{a}$ are the same,
so this identification respects pullback of sieves.
\end{remark}