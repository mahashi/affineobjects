\begin{lemma}\label[lemma]{9f0}
Let $\cat{C}$ be a category.
Let $a,b \in \cat{C}$.
Let $f:b \rightarrow a \in \overcat{C}{a}$.
We have the equalities $L^fQ^f = \id$ and $Q^fL^f = \id$.
\end{lemma}
\begin{proof}
Let $w: c \rightarrow a$.
Let $g:w \rightarrow f \in \overcat{C}{a}$.

Let $S\in \sievecat{f}$.
Let $h\in Q^fL^f(S)(g)$.
Hence $g=fh$ and $h\in L^f(S)(c)$.
This implies $h\in S(fh) = S(g)$.
Let $h\in S(g)$.
So $g=fh$ and $h \in L^f(S)(\dom{g})= L^f(S)(c)$.
This implies $h\in Q^fL^f(S)(g)$.
Therefore $Q^fL^f(S)$ and $S$ are the same sieve.

Let $h:S \rightarrow R \in \sievecat{f}$.
Let $p \in S(g)$.
Then by construction $L^fQ^f(h)_g(p) = Q^f(h)'_{c}(p) = h_{c}(p)$.

Let $R\in \sievecat{b}$.
Let $h\in L^fQ^f(R)(c)$.
Hence $h\in Q^f(R)(g)$ for some $g:c\rightarrow a$.
So $g=hf$ and $h\in R(c)$.
Let $h\in R(c)$.
Hence $h\in Q^f(R)(hf)$ and since $\dom{hf} = c$ we get $h\in L^fQ^f(R)(c)$.
Therefore $L^fQ^f$ and $R$ are the same sieve.

Let $h:S \rightarrow R \in \sievecat{b}$.
Let $p \in S(c)$.
Then by construction $Q^fL^f(h)_{c}(p) = L^f(h)_{pf}(p) = h_{c}(p)$.

So $L^fQ^f = \id$ and $Q^fL^f = \id$.
\end{proof}