%Categories&topologies remark 2.27
\begin{definition}[Induced relative topology]
Let $\top{T}{u(f)}$ be the set of covering sieves on $u(f) \in a$.
By the previous lemma sieves on $u(f)$ are sieves on $f$.
Let $\top{S}{f} = \{R \mbox{ sieve on f} \suchthat u(R) \in \top{T}{u(f)}$ be the induced topology.
So $u(R)$ is covering on $u(f)$ \iff $R$ is covering on $f$.

a) Since u commutes with pullback of sieves, 
we have $\maxsieve{u(f)} = u(\maxsieve{f}) = \maxsieve{f}$,
hence $\maxsieve{f} \in \top{S}{f}$.

b) Let $R$ be a covering sieve on $f$.
Let $h:b'\rightarrow a$
and $p:b' \rightarrow b$ with $fp = h$.
Commutativity of $u$ and pulling back implies that $u(p)^*u(R) = u(p^*R)$.
Hence $p^*R$ is covering since $u(p^*R)$ is.

c) Let $R$ be a covering sieve on $f$
and $Q$ be a sieve on $f$.
Let $h:b'\rightarrow a$
and $p:b' \rightarrow b \in R$,
hence with $fp = h$.
Assume $p^*Q$ is covering for every such $p$.
Then $u(p^*Q) = u(p)^*u(Q)$ is covering for every $p$.
We know that $u(R)$ is covering hence $u(Q)$ must be, which implies that $Q$ is covering.

We proved that $\top{S}$ is indeed a Grothendieck topology.
\end{definition}