\begin{lemma}[$\stilde$ commutes with restriction]
Let $\rsite{C}{T}{O}$ be a ringed site.
We have a natural isomorphism  
\[q:\direct{u} \circ \stilde \rightarrow \stilde \circ (\blank \tensor_{\sect{1}{O}} \sections{a}{O}).\]
\end{lemma}

\begin{proof}
Define $q$ to be the composition
\begin{align*}
	\direct{u} \circ \sheafify \circ \pstilde 
	& \xRightarrow{s^2\pstilde} \sheafify \circ \direct{u} \circ \pstilde \reason{lemma ?}\\
	& \xRightarrow{\sheafify{t}} \sheafify \circ \pstilde \circ \blank \tensor_{\sect{1}{O}} \sections{a}{O})\ \reason{lemma ?}
\end{align*}

From lemma ? and lemma ?, $t$ and $s^2$ are isomorphisms so $q$ is an isomorphism as well.
\end{proof}

% \begin{lemma}[$\stilde$ counit commute with restriction]
% Let $\rsite{C}{T}{O}$ be a ringed site.
% Let $a\in \cat{C}$.
% Let $\counit$ be the counit of the adjunction $\adj{\stilde}{\globalsections}$ on $\cat{C}$.
% Let $\counit_a$ be the counit of the adjunction $\adj{\stilde_a}{\globalsections}$ on $\overcat{C}{a}$.
% Let $M$ be a $\sect{a}{O}$-module.
% We have $\restr{\counit}{a} \iso \counit_a$ on modules of the form $\stilde{M}$.
% \end{lemma}

% \begin{proof}
% Fix $\stilde{M}$. 
% Let $N = \gblsect{F}$.

% Let $\counit_{\pstilde}$ be the counit of the adjunction $\adj{\pstilde}{\sections{1}}$.
% By lemma ?, restricting this counit yields the counit $\counit_{a, \pstilde}$ of the adjunction $\adj{\pstilde_a}{\sections{a}}$.
% We have the commuting diagram

% \begin{center}
% 	\begin{tikzcd}[row sep = large, column sep = large]
% 		\pstilde{N} \arrow{d}{\omega^2} \arrow{dr}{\counit_{\pstilde}} \\
% 		\stilde{N} \arrow{r}{\counit}
% 		& \stilde{M}
% 	\end{tikzcd}
% \end{center}

% After restriction and applying some natural isomorphisms, we have the commuting diagram
% \begin{center}
% 	\begin{tikzcd}[row sep = large, column sep = large]
% 		\pstilde{N \tensor_{\sect{1}{O}} \sect{a}{O}} 
% 			\arrow{d}{\omega^2_a} \arrow{drr}{\counit_{a,\pstilde}} \\
% 		\stilde{N \tensor_{\sect{1}{O}} \sect{a}{O}} \arrow[shift left]{rr}{\counit_a} 
% 		\arrow[shift right, swap]{rr}{\direct{u}(\counit)}
% 		&& \stilde{M \tensor_{\sect{1}{O}} \sect{a}{O} }
% 	\end{tikzcd}
% \end{center}

% By the universal property of $\omega^2_a$, it follows that $\inverse{q}\counit_a q = \restr{\counit}{a}$.
% \end{proof}

% \begin{lemma}
% Let $\site{C}{T}$ be a site.
% Let $a \in \cat{C}$.
% Consider the conunit $\counit$ of $\mainadjunction$ on $\cat{C}$
% and the counit $\counit_a$ of the same adjunction on $\overcat{C}{a}$. 
% Then $\restr{\counit}{a} = \counit_a$.
% \end{lemma}
% \begin{proof}
% Let $\module{F}$ be a sheaf module on $\cat{C}$.
% Let $f:b\rightarrow a\in \cat{C}$.
% We want to show that 
% $\counit_{\module{F},a,f}: \stilde{\globalsections{F}} \rightarrow \module{F}$
% is the same map as $\restr{\counit_{\module{F}}}{a}_{f}: \stilde{\globalsections{F}} \rightarrow \module{F} $
% \[\alpha:\pstilde(\globalsections{F}) \rightarrow \module{F},\]
% \[m\tensor r \mapsto mr\]
% composed with $\omega^2_{\pstilde{\globalsections{F}}}$.
% \end{proof}