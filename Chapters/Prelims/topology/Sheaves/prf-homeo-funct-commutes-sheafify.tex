\begin{lemma}\label[lemma]{b9f}
Let $F:\site{C}{T} \rightarrow \site{D}{S}$
be a morphism of sites and lift covers.
Then $\direct{F} \plus = \plus \direct{F}$ 
and hence also $\direct{F} \circ \sheafify = \sheafify \circ \direct{F}$.
\end{lemma}

\begin{proof}%TODO: fix this proof
% Let $c \in \cat{C}$.
% Let $R \in \cov{c}$.
% Let $\presheaf{G}$ be a presheaf on $\cat{D}$.
% Let $s = \mathfam{s}{i}{R}$ be a matching family in $\direct{F}\presheaf{G}$ 
% on $c$ indexed by $R$. So $s_i \in \sect{F(\dom{i})}{G}$.
% By the cover preserving assumption we know $F(R)$ generates a covering sieve $S$ on $F(c)$.
% Define the natural transformation $T: \direct{F}\plus \rightarrow \plus \direct{F}$ by 
% \[ s \mapsto \matchfam{t}{j}{S} \]

% where $t_j = F(h)s_i$ for some factorisation $j=F(i)h$. 
% Such a factorisation always exists since $S$ is generated by $F(R)$.
% This assignment is well-defined since $F$ preserves pullbacks, see proof of \cite[Lemma 2.3.3]{elephant}.

% We will prove that $T$ is injective.
% Let $s' = \mathfam{s'}{i}{R'}$.
% Assume $T(s)=T(s')$
% then $\matchfam{t}{j}{S} = \matchfam{t'}{j'}{S'}$,
% hence there exists covering sieve $Q \subset S\intersect S'$
% such that $t_k = t'_k$ for every $k\in Q$.
% Since $F$ lifts covers, there exists covering sieve $P\subset R\intersect R'$ on $c$
% such that $F(P)$ generates a sieve $P'\subset Q$.
% Hence $t_k = F(h)s_k = F(h)s'_k = t'_k$ with $k\in P'$ and $k=F(k)h$.
% Take $h=\id$ to conclude that $s=s'$ on $P$, hence they are in the same equivalence class.

% Now comes surjectivity.
% Let $\matchfam{t}{j}{S}\in..$ on $F(c)$.
% We get a covering sieve $R$ such that $F(R)\subset S$.


% Let $t = \matchfam{t}{j}{S}$ be a matching family on $F(c)$ with $S$ generated by $F(R)$
% for some covering sieve $R$ on $c$.
% The inverse of $T$ is the assignment
% \[ t \to \matchfam{s}{i}{R},\]

% where $s_i = t_{F(i)}$

% By cover lifting property we get a covering sieve $R$ such that $F(R) \subset S$.




% This is independent of the choice of $R$ because if $Q$ and $R$ are two different covering sieves
% with $F(Q)\subset S$ and $F(R) \subset S$ then $F(Q\intersect R) \subset S$
% These are clearly mutually inverses.
\end{proof}

\begin{corollary}
We have $\direct{F}\omega = \omega$ and hence also $\direct{F}\omega^2=\omega^2$
\end{corollary}

% \begin{proof}
% For any section $x \in \sections{b}{F}$.
% Let $x_i = \presheaf{F}(i)(x)$ for any morphism $i \in \cat{C}$. 
% Note that $\maxsieve{f} = \maxsieve{b}$.
% This implies that the compatible family $\set{x_i}$ indexed by the maximal sieve on $f$
% is sent by $s$ to the same set $\set{x_i}$ indexed by the maximal sieve on $b$.
% In diagram form, that

% \begin{center}
% 	\begin{tikzcd}
% 		\direct{u} \presheaf{F} \arrow{d}{\direct{u}\omega_{\presheaf{F}}} \arrow{dr}{\omega_{\direct{u} \presheaf{F}}} \\
% 		\direct{u} \plus{\presheaf{F}}  \arrow{r}{s} &
% 		\plus{(\direct{u} \presheaf{F})}
% 	\end{tikzcd}
% \end{center}

% commutes.
% \end{proof}


