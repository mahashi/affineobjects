% \begin{definition}[Restrictive functor]\label[definition]{Restrictive functor}
% A functor  $f: \rsite{C}{T}{O} \rightarrow \rsite{D}{S}{U}$ 
% between ringed sites is called restrictive 
% if for every quasi-coherent module $\module{G}$ on $\rsite{D}{S}{U}$
% the unit $\unit$ of $\adj{\inverse{f}}{\direct{f}}$ 
% induces an isomorphism
% \begin{align*}
% \unit_{\module{G}} :& \module{G} \rightarrow \direct{f}\inverseimg{f}\module{G},\\
% \unit_{\module{G}, \terminal} :& \globalsections{G} \rightarrow \globalsections{\direct{f}\inverseimg{f}G}\\
% \unit_{\module{G}, \terminal} \tensor_{\globalsections{O}} \globalsections{U} :
% 	&\globalsections{G} \tensor_{\globalsections{O}} \globalsections{U} 
% 	\rightarrow \globalsections{\direct{f}\inverseimg{f}G}.
% \end{align*}

% \end{definition}

\begin{definition}[Restrictive morphism]\label[definition]{Restrictive morphism}
Let $\rsite{C}{T}{O}$.
A morphism $f: a \rightarrow b \in \cat{C}$ is called restrictive 
if for every quasi-coherent module $\sheaf{G}$ on $\cat{C}$ the morphism
$$
\sect{b}{G} \tensor_{\sect{b}{O}} \sect{a}{O} \rightarrow \sect{a}{G}
$$
is an isomorphism.
\end{definition}	
