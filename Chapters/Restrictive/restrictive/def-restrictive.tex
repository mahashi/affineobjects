\begin{definition}[Restrictive morphism]\label[definition]{c94}
Let $\rsite{C}{T}{O}$.
A morphism $f: a \rightarrow b \in \cat{C}$ is called restrictive 
if for every quasi-coherent module $\sheaf{G}$ on $\overcat{C}{b}$ the morphism
\begin{equation}
\widehat{f}: \sect{b}{G} \tensor_{\sect{b}{O}} \sect{a}{O} \rightarrow \sect{a}{G}
\end{equation}\label[equation]{c98}
is an isomorphism.
\end{definition}

\begin{remark}\label[remark]{c9e}
Let us be in the setting of \Cref{c94}.
Assume $\sheaf{G} = \stilde{M}$ where $M=\gblsect{G}$.
Consider $\omega^2_{\sheaf{G},a}: \pstilde{M}(a) \rightarrow \stilde{M}(a)$.
This is map is equal to $\widehat{f}$.
\end{remark}