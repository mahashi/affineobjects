\begin{definition}[Restrictive functor]
A functor  $F: \rsite{C}{T}{O} \rightarrow \rsite{D}{S}{U}$ 
between ringed sites is called restrictive 
if for every quasi-coherent module $\module{G}$ on $\rsite{D}{S}{U}$
the co-unit $\counit$ of $\mainadjunction$ 
induces an isomorphism 
\[\counit_{\module{G}} : 
	\module{G} \rightarrow \direct{f}\inverse{f}\module{G},
\]
\[\counit_{\module{G}, \terminal} : 
	\globalsections{G} \rightarrow \globalsections{\direct{f}\inverse{f}G}
	\]
\[\counit_{\module{G}, \terminal} \tensor_{\globalsections{O}} \globalsections{U} :
	\globalsections{G} \tensor_{\globalsections{O}} \globalsections{U} \rightarrow \globalsections{\direct{f}\inverse{f}\module{G}}.
\]
\end{definition}

\begin{definition}[Restrictive morphism]
A morphism $f: a \rightarrow b \in \cat{C}$ is called restrictive 
if the induced functor 
\[\overcat{C}{a} \rightarrow \overcat{C}{b}\] is restrictive.
\end{definition}