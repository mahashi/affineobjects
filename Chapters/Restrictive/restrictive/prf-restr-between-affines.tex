\begin{lemma}\label[lemma]{3}
Any morphism $\Spec{S} \xrightarrow{f} \Spec{R} \in \overcat{Sch}{\Spec{R}}$ 
between affine schemes is restrictive.
\end{lemma}
\begin{proof}
Let $\sheaf{G}$ be a quasi-coherent module on $\overschemes{\Spec{R}}$.
Set $M = \sect{\Spec{R}}{G}$
We want to prove that 
\[
M \tensor_{R} S \rightarrow \sect{\Spec{S}}{G}
\]
is an isomorphism.
%TODO: note that this map is a component of \omega

Note that $\sheaf{G} = \stilde{M}$ since $\Spec{R}$ is affine. See \Cref{873}.
For the same reason, we get $\restr{\sheaf{G}}{\Spec{S}} = \stilde{M \tensor \sect{\Spec{S}}{O}}$.
%TODO: use the fact that restriction and stilde commute for the above fact.
By lemma ?, we know that $\omega^2$ is an isomorphism at affine schemes.

By lemma ?, we know that for a module of the form $\stilde{M}$ 
the $\omega^2$ at the codomain of $f$ is equal to $\hat{f}$
if both domain and codomain are affine schemes.
\end{proof}

%TODO: Define \hat{\blank} somewhere
%TODO: Proof that that $\omega^2$ is equal to $\hat{f}$ in certain situation
%TODO: remind reader what \omega is