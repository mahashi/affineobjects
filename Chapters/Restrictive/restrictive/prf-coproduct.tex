\begin{lemma}\label[lemma]{1}
Let $X,Y$ be a schemes.
The canonical morphism $X \rightarrow X \disjointunion Y$ is restrictive.
\end{lemma}

\begin{proof}
Let $\sheaf{G}$ be a quasi-coherent sheaf on $X \disjointunion Y$.
By the sheaf property $\sect{X \disjointunion Y}{G} = \sect{X}{G} \times \sect{Y}{G}$.
The same holds also for $\sheaf{O}$.

We are considering the morphism
\[
\sect{X \disjointunion Y}{G} \tensor_{\sect{X \disjointunion Y}{O}} \sect{X}{O} \rightarrow \sect{X}{G}.
\]

By the previous remark about disjoint unions and the sheaf property and some basic commutative algebra one sees that this becomes 
\[
(\sect{X}{G} \tensor \sect{X}{O}) \times (\sect{Y}{G} \tensor \sect{X}{O}) \rightarrow \sect{X}{G}.
\]

Since $\sect{Y}{G} \tensor \sect{X}{O} = 0$, we are left with

\[
(\sect{X}{G} \tensor \sect{X}{O}) \rightarrow \sect{X}{G}
\]
\[
g \tensor r \rightarrow rg.
\]

Note that $\sect{X}{G}$ already is an $\sect{X}{O}$-module and conclude that hence this morphism is an isomorphism.
\end{proof}

\begin{lemma}\label[lemma]{2}
If $X_i \rightarrow Y$ is restrictive for each $i\in I$, 
where $I$ is an (possibly infinite) indexing set,
Then $\Disjointunion_{i\in I} X_i \rightarrow Y$ is restrictive.
\end{lemma}
\begin{proof}
%TODO: proof
\end{proof}


\begin{lemma}[coproduct van restrictive]\label[lemma]{coproduct van restrictive}
Let $X_1,X_2,Y$ be a schemes.
$X_1 \rightarrow Y$ and $X_2 \rightarrow Y$ are restrictive morphisms \iff
the corresponding morphism $X_1 \disjointunion X_2 \rightarrow Y$ is restrictive.
\end{lemma}

\begin{proof}
Note that $\sect{X_1 \disjointunion X_2} = \sect{X_1} \times \sect{X_2}$ by the sheaf property.
We will show that
\[
\sect{Y}{G} \tensor_{\sect{Y}{O}} \sect{X_1 \disjointunion X_2}{O} \rightarrow \sect{X_1 \disjointunion X_2}{G}
\]
is an isomorphism.

Let $g \tensor (r,s) \in \sect{Y}{G} \tensor_{\sect{Y}{O}} \sect{X_1 \disjointunion X_2}{O}$.
This map sends it to $(rg,sg)$. 
By assumption the maps 
\begin{align*}
\sect{Y}{G} &\tensor_{\sect{Y}{O}} \sect{X_1}{O} \rightarrow \sect{X_1}{G}\\
 g &\tensor r \mapsto rg\\
\sect{Y}{G} &\tensor_{\sect{Y}{O}} \sect{X_2}{O} \rightarrow \sect{X_2}{G}\\
g &\tensor s \mapsto sg
\end{align*}
are isomorphisms. 
This implies directly that our map $g\tensor (r,s) \rightarrow (rg,sg)$ is an isomorphism.

\end{proof}