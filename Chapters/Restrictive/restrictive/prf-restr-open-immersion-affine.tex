\begin{lemma}[Restrictive to affines]\label[lemma]{e10}
If $f:X \rightarrow \Spec{R}$ is a restrictive open immersion,
then $X$ is affine.
\end{lemma}
\begin{proof}
Since $X$ is an open in $\Spec{R}$, 
we get a distinguised covering $\Union_i D(f_i) = X$ with $f_i\in R$
and $i\in I$.
We will prove that $(\restr{X}{f_i})_{i\in I} = (1)$ in $S = \sect{X}{O}$.

Then we invoke the result in \cite[Ex. 2.1.7]{harts} that states the following.
For a scheme $Y$ let $Y_f$  be the support of $f\in \sect{X}{O}$ as in \Cref{8d9}.
For a scheme $Y$ if $X_{g_j}$ are affine and $(g_j)_{j\in J}=(1)$
then $X$ is affine.

Note that $D(f_i) = X_{\restr{X}{f_i}}$.
Consider $M = \frac{R}{(f_i)}$ as an $R$-module and look at $\stilde{M}$.
By restrictiveness we get $M \tensor_R S = \stilde{M}(S)$ and  
by $M \tensor_R R_{f_i} = \stilde{M}(D(f_i)) = M_{f_i} = 0$. 
Hence $\stilde{M}(S) = 0$ by the sheaf axiom.
This implies that $(\restr{X}{f_i})_{i\in I} = (1)$ in $S$.
So $X$ is affine.
\end{proof}
