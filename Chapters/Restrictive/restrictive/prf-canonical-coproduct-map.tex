\begin{lemma}\label[lemma]{d46}
Let $X,Y$ be a schemes.
The canonical morphism $X \rightarrow X \disjointunion Y$ is restrictive.
\end{lemma}

\begin{proof}
Let $\sheaf{G}$ be a quasi-coherent sheaf on $X \disjointunion Y$.
By the sheaf property $\sect{X \disjointunion Y}{G} = \sect{X}{G} \times \sect{Y}{G}$.
The same holds also for $\sheaf{O}$.

We are considering the morphism
\[
\sect{X \disjointunion Y}{G} \tensor_{\sect{X \disjointunion Y}{O}} \sect{X}{O} \rightarrow \sect{X}{G}.
\]

By the previous remark about disjoint unions and the sheaf property and some basic commutative algebra one sees that this becomes 
\[
(\sect{X}{G} \tensor_{\sect{X \disjointunion Y}{O}} \sect{X}{O}) 
\times (\sect{Y}{G} \tensor_{\sect{X \disjointunion Y}{O}} \sect{X}{O}) \rightarrow \sect{X}{G}.
\]

Since $\sect{Y}{G} \tensor_{\sect{X \disjointunion Y}{O}} \sect{X}{O} = 0$, we are left with

\[
(\sect{X}{G} \tensor_{\sect{X \disjointunion Y}{O}} \sect{X}{O}) \rightarrow \sect{X}{G}
\]
\[
g \tensor r \rightarrow rg.
\]

Note that $\sect{X}{G}$ already is an $\sect{X}{O}$-module and conclude that hence this morphism is an isomorphism.
\end{proof}
