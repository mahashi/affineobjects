\begin{example}[Affine non-restrictive map]\label[example]{Affine non-restrictive map}
One might expect(or want) that any property of all maps between affine schemes
also hold for affine maps between any schemes.
This is not the case for restrictiveness.

Consider the canonical inclusions $\affspace{i} \rightarrow \projspace{i}$
and the shifted quasi-coherent module $\sheaf{O}(-1)$.
This module is even locally free of degree 1, this is often called an invertible module.

The global sections of the module $\sheaf{O}(-1)$ are the elements of degree $-1$
in the global sections of $\sheaf{O}$. There are no such elements, hence the global sections are the zero module.

On $\affspace{i}$ all invertible modules are isomorphic to the structure sheaf. See \cite{vakil} 14.2.8 %TODO: find way to reference items inside references
We conclude that the canonical inclusions cannot be restrictive.

Any inclusion $\Spec{\kappa(\Ip)} \rightarrow \projspace{i}$ of a point is not restrictive which can be shown with the same argument.

This is a (more opaque) way of saying that on projective space 
not every quasi-coherent sheaf is generated by global sections. 
\end{example}
