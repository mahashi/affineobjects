%% When is the canonical map A -> A\tensor B injective?
\begin{question}
Assume $\Spec A = U \subset V = \Spec B$ open affines of $X$.
Let $\shf$ be a quasi-coherent sheaf on $X$.
By the affine result, the restriction map $\shf(V) \rightarrow \shf(U)$ is injective \iff $\shf(V) \rightarrow \shf(V)\tensor_{\O_X(V)}\O_X(U)$ is injective.
Is $\shf(V) \rightarrow \shf(V)\tensor_{\O_X(V)} \O_X(U) \iso \shf(U)$ always injective?
\end{question}
\begin{Qanswer}
No.

Let $X=V$, $B= \Z$, $\shf= \widetilde{\Z/2\Z}$ and $U= D(2)$.
Then the localization map 

\[ \Z/2\Z \rightarrow (\Z/2\Z)_2 = 0\]
is not injective. 

\bigskip
When is it?
Let's go back to the general setting. 
Only let $\restr{\shf}{V}= \widetilde{M}$ and $U= D(f)$ for some $f \in B$.
Let $m\in M$. If $m \tensor 1 = 0$ in $M\tensor_B A$, then $f^n m = 0$ in $M$ for some $n\in \N$.
Hence the localization map $M\rightarrow M\tensor_B A$ is not injective \iff $f\in \sqrt{\Ann(m)}$ for some $m\in M$.
\end{Qanswer}
