\begin{example}[Examples of caffine objects]
The main example to keep in mind is $\Spec{R} \in \schemes$.
Let $\module{F}$ be a quasi-coherent module on $\Spec{R}$
via the presentation
\[\coproduct_{I} \restrict{U_i}{\sheaf{O}} \rightarrow \coproduct_J \restrict{U_i}{\sheaf{O}} 
	\rightarrow \restrict{U_i}{\module{F}} \rightarrow 0\]

for some cover $\cover{U_i}{\Spec{R}}$.
We know from lemma ? that $\module{F} = \tilde{\gblsect{F}}$.
Let $M$ be a $R$ module.

We will show that
\[\stilde{\gblsect{F}} \xrightarrow{\counit} \module{F}\]
and
\[M \xrightarrow{\unit} \gblsect{\stilde{M}}\]
\[m \mapsto \omega^2(m)\]
are isomorphisms.

Let $f \in \gblsect{O}$.
Note that 
\[\counit_{D(f)}: \sect{D(f)}{\stilde{\gblsect{F}}} \rightarrow \sect{D(f)}{\module{F}}\]
is the map
\[\gblsect{F}\tensor \sect{D(f)}{O} \rightarrow \gblsect{F}_f,\]
\[m \tensor r \mapsto mr.\]

This is an isomorphism by basic commutative algebra. Hence $\counit$ is an isomorphism, since domain and codomain are sheafs and $\counit$ is an isomorphism on the base of opens $D(f)$.

By lemma ? in Stacks, $\gblsect{\stilde{M}} = M$.
We have $\unit = \id_M$.
\end{example}