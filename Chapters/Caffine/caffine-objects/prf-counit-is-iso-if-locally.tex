\begin{lemma}\label[lemma]{1d9}
Let $\rsite{C}{T}{O}$ be a ringed site.
Let $a \in \cat{C}$.
Let $\cover{b_i}{a}$ be a caffine covering on $a$.
Assume every map $b_i \rightarrow a$ is restrictive.
Then the counit $\counit$ of the adjunction $\mainadjunction{a}$
is a natural isomorphism.
\end{lemma}


\begin{proof}
Let $\module{F}$ be a quasi-coherent sheaf module. 
Set $M = \sections{a}{F}$.
Set $M_i = \sections{b_i}{F}$.
Set $\beta_i = \restr{\counit_{\module{F},a}}{b_i}$.
By lemma ? $\beta_i \iso \counit_{b_i}$, hence $beta_i$ is an isomorphism.


% Since all the maps are restrictive, we have
% \[M\tensor_{\globalsections{O}} \sections{b_i}{O} \iso M_i.\]

% Consider the diagram 

% \begin{center}
%  	\begin{tikzcd}[row sep = large, column sep = large]
%  		\restr{\stilde{M}{a}}{b_i} \arrow{r}{\beta = \restr{\counit_{a}(\module{F})}{b_i}}
%  		& \restr{\module{F}}{b_i} \\
%  	\end{tikzcd}
%  \end{center}

% The map $\beta$ is a natural transformation between quasi-coherent modules over a caffine object, 
% hence is an isomorphism \iff the global component $\beta_{b_i}$ is.
% The global component $\beta_{b_i}$ is
% \[\beta':M \tensor_{\globalsections{O}} \sections{b_i}{O} \rightarrow M_i,\]
% \[m \tensor t \mapsto tm\]

% composed with the component at $b_i$ 
% of $i: \pstilde(M) \rightarrow \stilde(M) = \module{F}$.
% By lemma ?, the component at any caffine object 
% of the universal sheafification morphism for a quasi-coherent module is an isomorphism.
% We know $\beta'$ is an isomorphism because the map $b_i \rightarrow a$ is restrictive.
% Hence $\beta$ is an isomorphism.

% Consider the co-unit at $\module{F}$
% \[\counit_a(\module{F}): 
% 	\stilde{\sections{a}{F}} \rightarrow \module{F}.
% \]

% This morphism restricted gives
% \[\restr{\counit_{a}(\module{F})}{b_i}: 
% 	\restr{\stilde{\sections{a}{F}}}{b_i} \rightarrow \restr{\module{F}}{b_i},
% \]

% which is the same map as $\epsilon_{b}(\restr{\module{F}}{b})$.
% We only need to establish that $\restr{\stilde{\sections{a}{F}}}{b_i}$.

% Because $b_i$ is caffine, the canonical morphism given by sheafification
% \[ M \tensor_{\sections{a}{O}} \sections{b_i}{O} \rightarrow  M_i\] is an isomorphism.

% Hence the component 
% \[\epsilon_{b_i}(\restr{\module{F}}{b_i}): \stilde{\sections{b_i}{F}} \rightarrow \restr{\module{F}}{b_i}.\]
% is an isomorphism, because $\epsilon_{b_i}$ is a natural isomorphism because $b_i$ is caffine.
% The subscript ${b_i}$ signifies that we are working in $\cat{C}/b_i$.

% But over an caffine object, a map is an isomorphism \iff it is an isomorphism on global sections. 
% In this case, using naturality of $\epsilon$,
% \[\epsilon_{b_i}(b_i): \globalsections{F} \tensor_{\sections{a}{O}} \sections{b_i}{O} \rightarrow \sections{b_i}{F},\]
% \[ m \tensor r \rightarrow mr.\]

% By restrictiveness of the map $b_i\rightarrow a$, this map is an isomorphism. A local isomorphism between sheaves is an isomorphism.
\end{proof}



