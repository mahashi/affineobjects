\subsection{Caffine = Affine for schemes}

Let $(X,\sheaf{O})$ be a caffine scheme.
Let $X \xrightarrow{F} \Spec{\globalsections{O}} = Y$ be the adjunct of the identity map via the adjunction $(\Spec{\blank}, \globalsections{\blank})$.
Equivalently, let $F$ be the component at $X$ of the unit from this adjunction.

Let's introduce our variables.
Let $x,y\in X$. 
Let $\Ip,I,J\subset \sheaf{O}(X)$ be ideals with $\Ip$ prime.
Let $a,b\in \sheaf{O}(X)$ be global sections.

\begin{remark}
Recall that $F(x) = \ker(x)$ for $x\in X$.
I will use $D_Y(a)$ for the distinguished open defined by $a$ in the affine $Y$.
Note that $D_X(ab) = D_X(a) \cap D_X(b)$ since $\ker(x)$ is a prime ideal.
\end{remark}

\begin{remark}
If the support of a sheaf $\sheaf{G}$ is empty, then locally all sections are zero. 
Hence all sections are equal to the zero section and $\sheaf{G} = 0$.
\end{remark}




\begin{comment}
\begin{lemma}
Let $U_i\subset X$ be an affine open of $X$.
Then in the sequence $I\tensor \sheaf{O}(U_i)\xrightarrow{j}\Lambda_X(I)(U_i) \xrightarrow{i} \sheaf{O}(U_i)$ the map $j$ is an iso and the map $i$ is injective
\end{lemma}
\begin{proof}
Let $I\xrightarrow{g}\sheaf{O}(X)$ be the inclusion.
The map $i$ is the component at $U_i$ of $\Lambda_X(i)$, which is injective by left exactness of $\Lambda_X$.
The inclusion $U_i\subset X$ is restrictive because $U_i$ is affine, hence $j$ is an isomorphism.
\end{proof}

\begin{remark}
I will use both $I\tensor \sheaf{O}(U_i)$ and $I \sheaf{O}(U_i)$ to denote the image (ideal) of the composition of these maps.
\end{remark}

\begin{lemma} Let $K,L\subset \sheaf{O}(X)$ be ideals. Then
\[V_X(K)\subset V_X(L) \Rightarrow \sqrt{L}\subset \sqrt{K}.\]
\end{lemma}
\begin{proof}
Assume \[V_X(K)\subset V_X(L).\]
Let $U_i = \Spec A_i \rightarrow X$ be a family of affine opens that covers $X$.
Note that $V_X$ restricts to the familiar 'zero-locus' function $V(\cdot)$ on any affine open.
Hence on $U_i$, we get
\[V(K A_i)  \subset V(L A_i).\]

For any ideal $N\subset \sheaf{O}(X)$, we get $\sqrt{N (A_i)_d}=\sqrt{N} (A_i)_d$ for every $d\in A_i$.
This implies $\Lambda_{U_i}(\sqrt{N} A_i)= \Lambda_{U_i}(\sqrt{N A_i})$ as follows.
Let $a\in \sqrt{N A_i}$, then we can get a finite cover of $U_i$ by distinguished opens which provides a matching family of restrictions $a_d\in \sqrt{N (A_i)_d}=\sqrt{N} (A_i)_d$.
By the sheaf property $a\in \sqrt{N} A_i$.
We use this for
\[\sqrt{L A_i} = \sqrt{L} A_i \subset \sqrt{K A_i} = \sqrt{K} A_i\] 
as ideals of $A_i$.
Hence we get an injection
\[\restr{\Lambda_X(\sqrt{L})}{U_i} \xrightarrow{f_i} \restr{\Lambda_X(\sqrt{K})}{U_i}\] 
that respects the inclusions \[g:\Lambda_X(\sqrt{L}) \rightarrow \sheaf{O}\]
and \[h:\Lambda_X(\sqrt{K}) \rightarrow \sheaf{O}.\] 
This ensures moreover that $f_i = f_j$ on $U_{ij}$ because $h$ is a mono.
We get an injection
\[\Lambda_X(\sqrt{L}) \xrightarrow{f} \Lambda_X(\sqrt{K})\]
by glueing. Since $\Lambda_X$ is an equivalence, we get an injection of modules $\sqrt{L} \rightarrow \sqrt{K}$ 
that respects the inclusions $\sqrt{K}\subset \sheaf{O}(X)$ and $\sqrt{L}\subset \sheaf{O}(X)$.
\end{proof}
\end{comment}



\begin{comment}
\begin{definition}
Define $G(\Ip)$ to be the generic point of $V_X(\Ip)$.
\end{definition}
\begin{lemma}
The functions $F$ is a bijection.
\end{lemma}
\begin{proof}
Let $F(z)=\Ip$.
Note that $a\in F(G(\Ip))$ \iff $a\in \ker(x)$ for all $x\in V_X(\Ip)$.
We have
\begin{align*}
a\in \Ip &\Rightarrow a\in \ker(x), \forall x\in V_X(\Ip)\\
&\Rightarrow D_X(a) \cap V_X(\Ip) =\emptyset\\
&\Rightarrow a\in \ker(z)=\Ip\\
\end{align*}

Hence $FG = id_X$ and $F$ is surjective so a bijection.


Let $x\in X$. By construction $x\in V_X(F(x))$, because $F(x)=\ker(x)$.
Let $w\in V_X(F(x))$. Then $\ker(\sheaf{O}(X)\rightarrow \kappa(x))\subset \ker(\sheaf{O}(X)\rightarrow \kappa(w))$, so any open nhood $D_X(a)$ of $w$ contains $x$.
Therefore $x$ is the generic point of $V_X(F(x))$. Hence $GF = id_X$.


\end{proof}
\end{comment}





