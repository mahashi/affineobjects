Assume we have a restrictive caffine cover $\cover{b_i}{a}$. 
Let $\module{F}$ be a quasi-coherent sheaf module. 
Set $M = \sections{a}{F}$.
Set $M_i = \sections{b_i}{F}$ on $\cat{C}/a$.

Consider the co-unit at $\module{F}$
\[\epsilon_a(\module{F}): \Tildefunctor{\sections{a}{F}} \rightarrow \module{F}.\]

This morphism restricted gives
\[\epsilon_{a,b_i}(\module{F}): \restr{\Tildefunctor{\sections{a}{F}}}{b_i} \rightarrow \restr{\module{F}}{b_i},\]

which is the same map as $\epsilon_{b}(\restr{\module{F}}{b})$.
We only need to establish that $\restr{\Tildefunctor{\sections{a}{F}}}{b_i}$.

Because $b_i$ is caffine, the canonical morphism given by sheafification
\[ M \tensor_{\sections{a}{O}} \sections{b_i}{O} \rightarrow  M_i\] is an isomorphism.

Hence the component 
\[\epsilon_{b_i}(\restr{\module{F}}{b_i}): \Tildefunctor{\sections{b_i}{F}} \rightarrow \restr{\module{F}}{b_i}.\]
is an isomorphism, because $\epsilon_{b_i}$ is a natural isomorphism because $b_i$ is caffine.
The subscript ${b_i}$ signifies that we are working in $\cat{C}/b_i$.

But over an caffine object, a map is an isomorphism \iff it is an isomorphism on global sections. 
In this case, using naturality of $\epsilon$,
\[\epsilon_{b_i}(b_i): \globalsections{F} \tensor_{\sections{a}{O}} \sections{b_i}{O} \rightarrow \sections{b_i}{F},\]
\[ m \tensor r \rightarrow mr.\]

By restrictiveness of the map $b_i\rightarrow a$, this map is an isomorphism. A local isomorphism between sheaves is an isomorphism.




