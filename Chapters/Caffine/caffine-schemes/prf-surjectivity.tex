\begin{lemma}\label[lemma]{21b}
The map $\counit_X$ is surjective.
\end{lemma}
\begin{proof}
Let $\Ip\in \Spec R$ be a point in the target of $\counit_X$.
Then $\Lambda_X(\kappa(\Ip))$ is a quasi-coherent sheaf of modules.
In fact $\kappa(\Ip)\tensor_{\sect{X}{O}} \sheaf{O}(U)$ is a $\sheaf{O}(U)$ algebra 
for any open $U\subset X$, hence $\Lambda_X(\kappa(\Ip))$ is a quasi-coherent sheaf of algebras.
Hence we can compute the relative spec $\Rspec(\Lambda_X(\kappa(\Ip)))\xrightarrow{h} X$. 

Since $X$ is caffine we have $\Lambda_X(\kappa(\Ip))(X) = \kappa(\Ip)$.
So \[\Rspec(\Lambda_X(\kappa(\Ip)))\xrightarrow{\counit_X \circ h} \Spec{R}\]
factors as \[\Rspec(\Lambda_X(\kappa(\Ip)))\rightarrow \Spec{\kappa(\Ip)} \rightarrow \Spec{R}.\]
% Consider the $\Spec$ and $\sections{X}$ adjunction.
% We have
% \[\Rspec(\Lambda_X(\kappa(\Ip)))\rightarrow X \rightarrow \Spec R.\]
% This morphism corresponds to a morphism $R\rightarrow \kappa(\Ip)$ through the adjunction.
% This can factored and
% This can be factore is the canonical morphism $g:R\rightarrow \kappa(\Ip)$. 
% This morphism is also the adjunct of the composition
% \[\Rspec(\Lambda_X(\kappa(\Ip)))\rightarrow \Spec \kappa(\Ip)\rightarrow X,\]
% so both maps must be equal. 
This gives us a commutative square

\begin{center}
\begin{tikzpicture}[node distance=3cm, auto]
  \node (C) {$X$};
  \node (B) [right of =C] {$\Spec R$};
  \node (D) [below of= C]{$\Rspec(\Lambda_X(\kappa(\Ip)))$};
  \node (A) [below of= B]{$\Spec \kappa(\Ip)$};
  \draw[->] (C) to node {$\counit_X$} (B);
  \draw[->] (A) to node [swap] {}  (B);
  \draw[->] (D) to node [swap] {} (A);
   \draw[->] (D) to node [swap] {} (C);
\end{tikzpicture}
\end{center}

Since $\Lambda_X(\kappa(\Ip))$ is not the zero sheaf hence the structure sheaf of $\Rspec(\Lambda_X(\kappa(\Ip)))$ is non-zero.
This implies that $\Rspec(\Lambda_X(\kappa(\Ip)))$ is not the empty scheme. 
Therefore the point $\Ip$ is in the image of $\Rspec(\Lambda_X(\kappa(\Ip)))\rightarrow \Spec{\kappa(\Ip)} \rightarrow \Spec{R}$,
hence also in the image of $\counit_X$.
\end{proof}