\begin{lemma}\label[lemma]{386}
The sets $D_X(a)$ form a basis for the topology of $X$,
with $a\in R$.
\end{lemma}
\begin{proof}
Let $x\in D_X(a)$. Then $a\not\in\ker{x}$, hence $\frac{1}{a}$ exists in $\sheaf{O}_x$.
We get an open neighbourhood $V$ of $x$ such that $\frac{1}{a} \in \sect{V}{O}$.
Let $y\in V$, then clearly $a\not\in\ker{y}$. We have shown that $x\in V\subset D_X(a)$,
hence $D_X(a)$ is open.

Let $U\subset X$ be any open. Let $x\in U$.
By \Cref{395} we get $I$ such that $V_X(I) = \compl{U}$.
It follows that $x\not\in V_X(I)$ and $I\not\subset \ker(x)$ by \Cref{42b}.
So we get a $g\in I$ with $g\not\in \ker(x)$.
We get $x\in D_X(g)$ and $D_X(g) \subset U$, because $D_X(g) \intersect V_X(I) = \emptyset$
by \Cref{446}.

Let $a,b\in R$.
Note that $D_X(ab) = D_X(a) \cap D_X(b)$ since $\ker(x)$ is a prime ideal.
So the opens $D_X(a)$ form a basis.
\end{proof}