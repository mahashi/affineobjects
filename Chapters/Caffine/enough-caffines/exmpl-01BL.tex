\begin{example}\label[example]{465}
This is the example in \cite[\href{http://stacks.math.columbia.edu/tag/01BL}{Tag 01BL}]{stacks}.

Let $L = (\R, \sheaf{O}_R)$ be the real line with the euclidean topology 
and the sheaf of continuous real valued functions as structure sheaf.
Let \[X = \frac{\bigcup_{i=0}^\infty L_i}{\sim}\] 
with $[i,x] \sim [j,y]$ \iff $i=j$ and $x=y$ or $y=x=0$. 
The real lines are glued to each other at zero.
Define the open $U_n\subset X$ as $U_n\cap L_i=(-\frac{1}{n},\frac{1}{n})$. 
These opens form a basis of neighbourhoods of $0$.
Let $f:\R\rightarrow \R$ be any continuous function such that $f(x)=0$ 
if $x\in (-1,1)$ and $f(x)=1$ if $x\in (-\infty,-2)\cup (2,\infty)$.
Let $f_n(x)=f(nx)$.

%Define map
Define the sheaf map 
\[\coproduct_i \sheaf{O}_R \xrightarrow{\alpha} \coproduct_{ij} \sheaf{O}_R,\]
\[e_i \mapsto \sum_{j} f_i \indicator_{L_j} e_{ij}.\]

To proof that this is well-defined, we need to show that the sum $\sum_{j} f_i \indicator_{L_j} e_{ij}$ 
is locally finite for every $i$.
Let $[k,y]\in X$. If $y\neq 0$, then \[W_{[k,y]}=\{[k,z]\in X \mid z\in(y-\delta, y+\delta)\}\subset L_k\] is open in $X$
and $\alpha_{W_{[k,y]}}(e_i)  = f_ie_{ik}$ for any $\delta< \abs{\frac{y}{2}}$.
If $y=0$, then $\alpha_{U_n}(e_i) = 0$ if $n>i$ because $f_i$ is zero on $U_n$.
Hence we found a cover on which our sum is locally finite, which makes $\alpha$ well-defined.

%Domain&codomain are associated sheafs
The adjunction $\adj{\stilde}{\sect{X}}$ implies 
that $\stilde$ commutes with arbitrary colimits.
Moreover
\[\sheaf{O}_X \iso \stilde(\sect{X}{O}),\]

so

\[\coproduct_i \sheaf{O}_R \iso \stilde(\coproduct_i \sect{X}{O}).\]

This shows that $\alpha$ is a morphism between objects that are of the form $\stilde{M}$.

Let $\beta: \coproduct_i \sect{U}{O} \rightarrow \coproduct_{ij} \sect{U}{O})$ for some open $U$.
Then $\stilde(\beta)(e_i) = \sum_{j\in J_i} a_{ij} e_{ij}$ where $J_i$ is finite for every $i$.

%Contradiction
Assume that $\alpha = \stilde(\beta)$ over some neighbourhood $U$ of $0$.
Then there exists a $m$ such that $U_m\subset U$. Let $k>2m$.
Then $f_k\neq 0$ on $U_m$, hence $f_k \indicator_{L_j}\neq 0$ on $U_m$ for every $j$ and so no co{\"e}fficients vanish of $\alpha_{U_m}(e_k) =\sum_{j} f_k \indicator_{L_j} e_{kj}$. 
This contradicts $\alpha=\stilde(\beta)$. Hence $\alpha$ does not come from a module map locally.
\end{example}

\begin{remark}\label[remark]{495}
This is an example of a ringed space where $\stilde$ is not an equivalence locally,
since it shows that $\stilde$ is not full. We will categorify this example next.
\end{remark}