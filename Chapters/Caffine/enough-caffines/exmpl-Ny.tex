%%% Example moved to N(y) setting
Let $X,f_j$ and $U_n$ be as in the previous example.
Define the full subcategory $N(y)\xrightarrow{i}\opens(X)$ of all opens $U$ that contain the point $y\in X$.

This category has all fibre products. Let $U\rightarrow V \leftarrow W$ be two morphisms.
Then $U \leftarrow U\cap W \rightarrow W$ is the pullback.

On this category $N(y)$, let a family $\{f_i:U_i\rightarrow U\}$ be covering if $\union_i f_i(U_i) = U$. 

Let $V\xrightarrow{f} U$ be an isomorphism, then $f(V)=U$ so $\{f\}$ is a covering family.

Let $\{U_i \xrightarrow{f_i} U\}$ be a covering family. 
For every $i$, let $\{U_{ij}\xrightarrow{f_{ij}} U_i\}$ be a covering family. 
By definition this gives that $\union_i f_i(U_i)=U$ and $\union_j f_{ij}(U_ij)=U_i$ for every $i$. 
Hence \[\union_{i,j} (f_i \circ f_{ij}) (U_{ij}) = \union_{i} f_i(U_i)=U\] and so the family 
$\{U_{ij}\xrightarrow{f_i \circ f_{ij}} U\}$ is covering.

Let $V\rightarrow U$ be a morphism in $N(y)$ and $\{U_i \xrightarrow{f_i} U\}$ be a covering family on $U$.
This tells us that $\union_i f_i(U_i)=U$, hence also $\union_i g_i(U_i\cap V) = V$ where $g_i:U_i\cap V\rightarrow V$ is the pullback of $f_i$. 
Hence $\{U_i\cap V\xrightarrow{g_i}V\}$ is a covering family of $V$.

All criteria for a pretopology are established. Let $\tau$ be the generated Grothendieck topology.

%%i is continuous
Let $\sheaf{F}$ be a sheaf on $\opens{X}$.
Let $\hat{\sheaf{F}} = \sheaf{F}\circ i$.
Let $\cover{U_i}{V}$ be a covering family on $V$ in $N(y)$.
Let $(x)_i$ be a matching family of $\hat{\sheaf{F}}$ indexed by $\cover{U_i}{V}$, so $x_i \in \hat{\sheaf{F}}(U_i) = \sheaf{F}(U_i)$.
Note that $\cover{U_i}{V}$ is also a covering family on $V$ in $\opens{X}$, hence $(x)_i$ is also a matching family of $\sheaf{F}$ on $V$.
Since $\sheaf{F}$ is a sheaf, there exists a unique amalgamation $x\in \sheaf{F}(V)=\hat{\sheaf{F}}(V)$ such that $x=x_i$ in $\sheaf{F}(U_i)=\hat{\sheaf{F}}(U_i)$. 
This shows that $\hat{\sheaf{F}}$ is a sheaf, hence $i$ is continuous.

Let $\sheaf{O}_{X,y} = \sheaf{O}_X\circ \tau$. This is a sheaf of rings by the previous.
We constructed a ringed site $(N(y),\tau,\sheaf{O}_{X,y})$.

%%adjunctions
Let $F$ be the inclusion functor from the category of sheafs to the category of presheafs. 
We have the adjunctions:

\begin{enumerate}
\item $(\Lambda(-),\Gamma(X,-))$,
\item $((-)^{++},F)$.
\end{enumerate}

%%coproduct is still associated, holds in every ringed site
The structure sheaf $O_{X,y}$ is, trivially, isomorphic to $\Lambda(\Gamma(X,O_{X,y}))$.
By adjunction (2)
\[\coproduct_i \sheaf{O}_{X,y} \iso (\coproduct_i \sheaf{O}_{X,y})^{+s},\]
where the coproduct on the left hand side is in the category of sheafs and the coproduct on the right hand side in the category of presheafs.

By adjunction (1)
\[\coproduct_i \Lambda(\Gamma(X,O_{X,y})) \iso \Lambda(\coproduct_i \Gamma(X,O_{X,y})).\]

Combine these 3 observation to get

\[\coproduct_i \sheaf{O}_{X,y} \iso \Lambda(\coproduct_i \Gamma(X,O_{X,y})),\]

which shows that $\coproduct_i \sheaf{O}_{X,y}$ is quasi-coherent.


%%Define \alpha
Set $y=0\in X$.
Define the sheaf map 
\[\coproduct_i O_{X_y} \xrightarrow{\alpha} \coproduct_{ij} O_{X_y},\]
\[e_i \mapsto \sum_{j} f_i \indicator_{L_j} e_{ij}.\]

%%Proof \alpha is well-defined
Fix $i$. We will prove that $\alpha_X(e_i)$ is a well-defined global section.
Let $m>i$. Let $V_k=L_k\cup U_m$ and $\cover{V}=\{V_k\}$.
By construction $f_i$ is zero on $U_m$, hence $f_i \indicator_{L_j}$ is zero on $V_k$ if $k\neq j$ and so $\sum_{j} f_i \indicator_{L_j} e_{ij} = f_i\indicator{L_k}e_{ik}$ on $V_k$.
This shows that $\alpha_X(e_i)$ is a well-defined section on any element of the cover $\cover{U_i}{V}$ and
this family is matching since the sections are functions and the 'restriction' maps are actual restriction.

%%Absence of a \beta for every element
Assume there exists $\beta: \coproduct_i \Gamma(V,O_{X,y}) \rightarrow \coproduct_{ij} \Gamma(V,O_{X,y})$ such that $\Lambda(\beta)=\alpha_V$.
Then $\alpha_V(e_i)=\sum_{j} f_i \indicator_{L_j} e_{ij}$ is not just locally finite over some cover, but actually finite globally on $V$ for all $i$.
So almost all $f_i \indicator_{L_j}$ are zero on $V$. 
Note that $y\in V$, so $U_d\subset V$ for some $d$. 
Let $i>2d$, then $f_i\neq 0$ on $(-\frac{1}{d},\frac{1}{d})$ and so $f_i \indicator_{L_j}\neq 0$ on $U_d$ for any $j$.
Hence $\alpha_V(e_i)=\sum_{j} f_i \indicator_{L_j} e_{ij}$ is not a finite sum for $i>2d$. 
This contradicts our assumption.

%%Mistake
Let $U\subset X$ be $U\cap L_j = U_{j}$. Fix $i$.
Then $f_i\indicator_{L_j}=0$ if $i<j$, hence \[\sum_{j} f_i \indicator_{L_j} e_{ij}=\sum_{j\leq i} f_i \indicator_{L_j} e_{ij}\] is a finite sum.

%%Conclude no element is affine
The restriction of any quasi-coherent sheaf is quasi-coherent. 
Observe that $\alpha$, and its restrictions, is a morphism between quasi-coherent sheafs but does not come from a map of modules.
Therefore $\Lambda(-)_V: \Gamma(V, O_{X,y})-\mbox{Mod} \rightarrow \qcoh{V}$ is not full for any $V$ and no object $V$ is affine in $N(y)$.
