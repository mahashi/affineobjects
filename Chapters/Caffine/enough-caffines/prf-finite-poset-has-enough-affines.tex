\begin{lemma}\label[lemma]{409}
Any ringed site $\rsite{C}{T}{O}$ that has a poset as underlying category, has enough caffines.
\end{lemma}
\begin{proof}
Let $x_0\in C$. 
If $x_0$ is covered by the maximal sieve only or the maximal sieve and the empty sieve, it is affine and we are done. 
Assume otherwise.
Let $S = \{y_i \rightarrow x_0\}$ be a non-maximal, non-empty cover of $x_0$.
Then $S$ does not contain isomorphisms.

We can associate to any non-maximal non-empty covering sieve $S$ of an element $x_0$,
the set of all NA-chains $x_0\leftarrow x_1 \leftarrow \ldots \leftarrow x_n$.
An NA-chain, associated to $R$, is a chain of maps ending in $x_0$ such that $x_i \leftarrow x_{i+1}$ is contained in a non-maximal, 
non-empty cover of $x_i$, where $x_0\leftarrow x_1$ is contained in $R$.

By finiteness of $C$, any chain of maps is bounded by the size of $C$ or contains a cycle. 
If a chain contains a cycle, it contains isomorphisms. 
By construction, no isomorphism can be present in a NA-chain. 
Therefore the length of any NA-chain is bounded by $\card{C}$.

Let $H$ be a NA-chain associated to $S$ of maximal length $m$. 
Then the last map $\ldots \leftarrow h \leftarrow g$ in $H$ has an affine object $g$ as domain,
because $H$ cannot be increased and so $g$ has no non-maximal, non-empty coverings which makes it affine.
Also the non-maximal, non-empty covering of $h$ where this map appears must be an affine covering by applying the same reasoning to the other objects occurring in it. 
Hence all objects occurring at the $(m-1)$th place in any NA-chain admits an affine cover.
Let $i\leq m-1$. Assume all elements at the $(i-1)$th place admit an affine cover. 
Let $b$ be a object occurring at the $(i-1)$th place in a chain. It is either affine or all objects in any non-maximal, non-empty cover occur at the $i$th place in some chain hence admit an affine cover. Therefore any non maximal, non empty cover on $b$ can be refined to an affine cover. This provides us with an affine cover of $b$.
By reversed induction, $x_0$ admits a affine cover.
\end{proof}
