\subsection{introduction}

Let $\ringedsite{C}{T}{O}$ be a ringed site.

\begin{definition}[Caffine]
Let $a\in \cat{C}$ be an object. 
We call $a$ \emph{caffine} if the adjunction $\mainadjunction$ is an equivalence of categories. 
Or equivalently that the unit $\unit$ and co-unit $\counit$ of this adjunction are natural isomorphisms.
\end{definition}

\begin{example}[Examples of caffine objects]
The main example to keep in mind is $\Spec{R} \in \schemes$.

% The only sheaf on a $\cat{C}/a$ for an object $a$ covered by the empty sieve is the terminal presheaf.
% For example the category $\opens{X}/\emptyset$ for some topological space $X$. 
% Note that $\emptyset$ has 2 covering sieves, the maximal sieve and the empty sieve.

% Any presheaf is a sheaf on a category localized at a object covered only by the maximal sieve. 

Let $\ringedspace{\pt}{R}$ be a ringed space. This space is always caffine, because all presheaves are sheaves.
If $R$ is non-local, then this space is not a scheme. This is an example of a non-scheme caffine ringed space
\end{example}