\begin{theorem}[Morphism between caffines is restrictive]
Let $Y = \rsite{C}{T}{O}$.
Let $f: b \rightarrow a \in \cat{C}$ be a morphism between caffine objects, 
then $f$ is restrictive.
\end{theorem}

\begin{proof}
Let $\module{F}$ be a quasi-coherent module on $\oversite{Y}{a}$.
Let $M = \sections{a}{F}$.
Since $a$ is caffine, we have $\sheaf{F} = \stilde{M}$.

We have to show that the adjunct of $f$
\[\sections{a}{F} \tensor_{\sections{a}{O}} \sections{b}{O} \rightarrow \sections{b}{F}\]
is an isomorphism.


%owen-caff-restr.tex
Consider
\begin{center}
 	\begin{tikzcd}[row sep = large, column sep = large]
		\pshmodules{\oversite{Y}{a}} \arrow{r}{\sheafify_a}\arrow{d}{\restrictOpt{b}} 
		& \shmodules{\oversite{Y}{a}} \arrow{d}{\restrictOpt{b}}
		& \modules{\sections{a}{O}} \arrow{l}{\stilde} \arrow[bend right]{ll}{\pstilde} 
		\arrow{d}{\blank \tensor \sections{b}{O}}\\
		\pshmodules{\oversite{Y}{b}} \arrow{r}{\sheafify_a}
		& \shmodules{\oversite{Y}{b}}
		& \modules{\sections{b}{O}} \arrow{l}{\stilde}\arrow[bend left]{ll}{\pstilde}

 	\end{tikzcd}
 \end{center}

By a previous lemma, the left square commutes. 
By definition the two `triangles' commute too and the outer square commute,
hence the right square also commutes. 
Therefore $M \tensor \sections{b}{\sheaf{O}} \iso \sections{b}{\sheaf{F}}$.
This is the proof you wrote down friday.

The requirement is not to find any isomorphism but a specific one.
So I think this is not enough and we need to do some bookkeeping and see if the witnessing isomorphism is our map.

Let $i$ be the morphism of presheaves at $\pstilde(M)$ of the natural transformation $\omega^2$ coming with $\sheafify_a$ as defined in lemma ?.
Let $j$ be the morphism at $\pstilde( M \tensor \sections{b}{O})$ of the natural transformation $\omega^2$ coming with $\sheafify_a$ as defined in lemma ? .

%my-caff-restr.tex
Consider
 \begin{center}
 	\begin{tikzcd}[row sep = large, column sep = large]
 		\restrictOpt{b}{\stilde{M}}
 		& \stilde( M \tensor \sections{b}{O}) \arrow{l}{s_{\pstilde(M)}} \\
 		\pstilde( M \tensor \sections{b}{O}) \arrow{u}{\restrictOpt{b}{i}} \arrow{ru}{j}
 	\end{tikzcd}
 \end{center}


We have seen that the component $j_b$ at $b$, the global component, 
is an isomorphism in lemma ?. since $b$ is caffine 
and that $s_{\pstilde(M)}$ is an isomorphism as constructed in lemma ?.

We will prove commutativity of the triangle.
Let $g: c \rightarrow b \in \oversite{Y}{b}$.
Let $\presheaf{M} = \pstilde( M \tensor \sections{c}{O})$.
Let $x = m \tensor r \in \presheaf{M}$.

- TODO% $\omega^2$ creates 2 dimensional( indexed by 2 sieves) families
% Then $j_g(x) = (\presheaf{M}(i)x)_{i \in \max{g}}$ indexed by the maximal sieve on $g$
% and $i_g(x) = i_{fg}(x) = (x_{i})$ indexed by the maximal sieve on $gf$.
% Hence we get  $s_g(j_g(x)) = i_g(x)$, so the triangle commutes.

Evaluating everything on the terminal object, in this case on $b$, 
shows that two out of three maps are isomorphisms, hence $i_b$ is an isomorphism.
\end{proof}

% The map $k$ is the adjunct of $\module{F}(f)$ with respect to the adjunction between restricting scalars and extending scalars along the map 
% $\sections{a}{O} \rightarrow \sections{b}{O}$. 
% More concretely, this map is

% \[k: x \tensor m \mapsto \module{F}(f)(x)m.\]

% The argument will go as follows. 
% First we observe that the morphism $\counit_{\module{F}}:\module{F} \rightarrow \stilde{\globalsections{F}}$ is an isomorphism because $a$ is caffine.
% Second $i_{a}:\Kappa{\globalsections{F}}(a) \rightarrow \stilde{\globalsections{F}}(a)$ is an isomorphism by lemma .. . 
% This holds for any caffine objects, so also for $b$. 
% The consequence is that $\Kappa{\globalsections{F}}(f) = i_b^{-1} \module{F}(f) \circ i_a$, by naturality of the transformation 
% $i: \Kappa{\globalsections{F}}\rightarrow \module{F}$.
% Third, show that $\Kappa{\globalsections{F}}(f)$ has an isomorphism as adjunct along the same extension/restriction adjunction. Call this adjunct $k'$.
% Fourth, use naturality of adjunction bijections to conclude that $k$ must also be an isomorphism.

% Since $a$ is caffine, $\module{F} = \stilde{\globalsections{F}}$.
% Since $\stilde{\blank} = \sheafify{\blank} \circ \Kappa{\blank}$, we know that $\module{F}$ is the sheafification of the presheaf 
% \[\Kappa{\globalsections{F}} = c \rightarrow \sections{a}{F} \tensor_{\sections{a}{O}} \sections{c}{O}.\]
% Set $M = \sections{a}{F}$.

% Define $k': \sections{a}{\Kappa{M}} \tensor_{\sections{a}{O}} \sections{b}{O} \rightarrow \sections{b}{\Kappa{M}}$
% By $k':x \tensor m \mapsto \Kappa{M}(f)(x) m$.
% If you unfold the constructions, it follows that 
% \[k'(x\tensor m) = x\tensor m \in \sections{b}{\Kappa{M}} = M \tensor \sections{b}{O}\]
%  is actually the identity.

% We will prove that $
% A) The component at a caffine object of the unversal sheafification morphism $i$ is an isomorphism
% As is stated in lemma .. , we have $\globalsections{i} = \counit_{T,a}$. 
% Hence when $a$ is caffine then $\globalsections{i}$ on is an iso.
% Note that $\globalsections{i}:\cat{C}/x \rightarrow \modules{\sheaf{O}(x)}$ is equal to $\sections{x}{i}: \cat{C}{y} \rightarrow \

% %By naturality of the universal sheafification transformation $i$, $\sections{a}{i}$ is an iso for all caffine $a$. 


% B) Adjunction bijection respects composition with isos
% We have now that $\Kappa{M}(f) = i_b^{-1} \circ \stilde{F}(f) \circ i_a$. Let $F$ be the bijection from the adjunction $\mainadjunction$.
% Then $

% C) Hence $k$ is also an iso.



% Note that restricting and sheafification commute.%TODO: explain
% We can first restrict our presheaf to $\cat{C}/b$ and then sheafify. 
% The global sections component of the universal sheafification morphism will be 
% \[M \tensor \sections{b}{O} \rightarrow \sections{b}{F},\]
% \[m \tensor r \mapsto mr\]

% because the triangle 

% \begin{center}
% \begin{tikzpicture}[node distance=2cm, auto]
%   \node (C) {$\sections{a}{F}$};
%   \node (B) [right of =C] {$\sections{b}{F}$};
%   \node (D) [below of= C]{$\sections{a}{F}\tensor \sections{b}{O}$};
%   \draw[->] (C) to node {$\mbox{restr}$} (B);
%   \draw[->] (D) to node [swap] {}  (B);
%   \draw[->] (C) to node [swap] {} (D);
% \end{tikzpicture}
% \end{center}

% must commute by naturality. This is exactly the component of the unit $\eta$ of the $\mainadjunction$ on $\site{C}{T}{O}/b$ for $\sections{a}{F} \tensor \sections{b}{O}$.
% Since $b$ is caffine, this is an isomorphism by assumption.

% Let $\rsite{C}{T}{O}$ be a ringed site.
% Let $R=\globalsections{O}$. Let $a\in C$ and $f:x\rightarrow y \in C$.