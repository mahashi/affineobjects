\subsection{Caffine = Affine for schemes}

Let $(X,\sheaf{O})$ be a caffine scheme.
Let $X \xrightarrow{F} \Spec{\globalsections{O}} = Y$ be the adjunct of the identity map via the adjunction $(\Spec{\blank}, \globalsections{\blank})$.
Equivalently, let $F$ be the component at $X$ of the unit from this adjunction.

Let's introduce our variables.
Let $x,y\in X$. 
Let $\Ip,I,J\subset \sheaf{O}(X)$ be ideals with $\Ip$ prime.
Let $a,b\in \sheaf{O}(X)$ be global sections.

\begin{definition}
Define \[\ker(x)=\ker(\sheaf{O}(X)\rightarrow \kappa(x)),\]
\[D_X(a)=\{x\in X \st a\not\in \ker(x)\},\]
Define \[V_X(I) = \Sup(\Lambda_X(\frac{\sheaf{O}(X)}{I})).\]
\end{definition}

\begin{remark}
Recall that $F(x) = \ker(x)$ for $x\in X$.
I will use $D_Y(a)$ for the distinguished open defined by $a$ in the affine $Y$.
Note that $D_X(ab) = D_X(a) \cap D_X(b)$ since $\ker(x)$ is a prime ideal.
\end{remark}

\begin{remark}
If the support of a sheaf $\sheaf{G}$ is empty, then locally all sections are zero. Hence all sections are equal to the zero section and $\sheaf{G}=0$.
\end{remark}

\begin{lemma}
The set $V_X(I)$ is closed.
\end{lemma}

\begin{proof}
Let $z\in X$ and $M$ a $\sheaf{O}$-module. Assume $z$ is in the support of $M$, then $g\neq 0$ for any generating element $g\in M_z$. 

Consider the exact sequence 
\[\sheaf{O}(X)\rightarrow \frac{\sheaf{O}(X)}{I} \rightarrow 0.\]
The functor $\Lambda_X$ is a left adjoint hence right exact so 
\[\sheaf{O} \xrightarrow{f} \Lambda_X(\frac{\sheaf{O}(X)}{I}) \rightarrow 0\]
is exact.
Hence the sequence
\[\sheaf{O}_x \xrightarrow{f_x} \Lambda_X(\frac{\sheaf{O}(X)}{I})_x \rightarrow 0\]
is exact. 
The global section $f(1)$ must generate $\Lambda_X(\frac{\sheaf{O}(X)}{I})$ as a module by surjectivity of $f$.
Similarly $f_x(1_x)$ generates $\Lambda_X(\frac{\sheaf{O}(X)}{I})_x$.

Note that $f_x(1_x) = f(1)_x$ by definition of $f_x$, hence $f(1)_x$ is a generating element.
Hence $\Lambda_X(\frac{\sheaf{O}(X)}{I})_x \neq 0$ \iff $f(1)_x \neq 0$. 

This implies $V_X(I) = \Sup(f(1))$ which makes $V_X(I)$ closed as the support of a global section.
\end{proof}


The functor $\Lambda_X$ is exact, so it commutes with quotients.
So
\[\Lambda_X(\frac{\sheaf{O}(X)}{I}) = \frac{\sheaf{O}}{\Lambda_X(I)}\]
and 
\[\Lambda_X(\frac{\sheaf{O}(X)}{I})_x = \frac{\sheaf{O}_x}{\Lambda_X(I)_x} = \frac{\sheaf{O}_x}{I\tensor \sheaf{O}_x} \]

$\frac{\sheaf{O}_x}{\Lambda_X(I)_x} \neq 0$, which is the same as saying that  $\Lambda_X(I)_x$ is a proper ideal of $\sheaf{O}_x$. 
The sheaf $\Lambda_X(I)_x$ is the sheafification of the presheaf $(U \mapsto I \tensor \sheaf{O}(U))$, hence the stalk at $x$ of the sheaf is 
$\colim{x \in U} I\tensor \sheaf{O}(U)$. The functor $I\tensor -$ is a left adjoint, hence commutes with colimits.
So the stalk is isomorphic to $I\tensor \colim{x\in U} \sheaf{O}(U)= I\tensor \sheaf{O}_x$. See Stacks[01BH]. 

\begin{lemma}
For $x\in X$ TFAE:
\begin{enumerate}
\item $x\in V_X(I)$
\item $I\sheaf{O}_x\neq \sheaf{O}_x$ 
\item $I\subset \ker(x)$.
\end{enumerate}
\end{lemma}
\begin{proof}
1 $\Rightarrow$ 2:

Assume $x\in V_X(I)$. 
Then $\Lambda_X(\frac{\sheaf{O}(X)}{I})_x = \frac{\sheaf{O}_x}{I\sheaf{O}_x} \neq 0$. 
Hence $I\sheaf{O}_x \neq \sheaf{O}_x$.

2 $\Rightarrow$ 3: 

Assume $I\sheaf{O}_x\neq \sheaf{O}_x$. 
Then $I\sheaf{O}_x$ is proper hence contained in the unique maximal ideal of the local ring $\sheaf{O}_x$, 
therefore $I\mapsto 0$ in $k(x)$ or equivalently $I\subset \ker(x)$.

3 $\Rightarrow$ 1:

Assume $I \subset \ker(x)$. Then $I$ maps into $\Im_x$, hence $I\sheaf{O}_x \subset \Im_x$. Therefore 
\[\frac{\sheaf{O}_x}{\Lambda_X(I)_x} = \frac{\sheaf{O}_x}{I\sheaf{O}_x} \neq 0.\]

\end{proof}
\begin{corollary}
If $y\in I$ then $D_X(y)\cap V_X(I)= \emptyset$ 
\end{corollary} 
\begin{proof}
Assume $y\in I$.
Let $z\in V_X(I)$, then $y\in \ker(z)$ by the previous lemma.
This implies $z\not\in D_X(y)$
\end{proof}
\begin{corollary}
$V_X(I)\cup V_X(J) = V_X(IJ)$
\end{corollary}
\begin{proof}
Let $z\in V_X(I)\cup V_X(J)$. Then $I\subset \ker(z)$ and $J\subset \ker(z)$ by the lemma, hence $IJ\subset \ker(z)$.
Apply the lemma again to get $z\in V_X(IJ)$.
Let $z\in V_X(IJ)$. Then $IJ \subset \ker(z)$ by the lemma. The ideal $\ker(z)$ is prime, so $I\subset \ker(z)$ or $J\subset \ker(z)$. Invoke the lemma again to get $z\in V_X(I)\cup V_X(J)$.
\end{proof}


\begin{lemma}
Every closed set $W$ can be written as $V_X(I)$ for some ideal $I$.
\end{lemma}
\begin{proof}
Let $\sheaf{I}$ be some ideal sheaf inducing a closed subscheme structure on $W$.
Let $\sheaf{O}_W$ be the structure sheaf of this closed subscheme.
By construction $V_X(I)$ is the support of the push-forward of $\sheaf{O}_W$, hence $V_X(I) = W$.
\end{proof}

\begin{lemma}
The sets $D_X(a)$ form a basis for the topology of $X$.
\end{lemma}
\begin{proof}
Let $U\subset X$ be any open. Let $x\in U$.
By the previous lemma we get $I$ such that $V_X(I)=U^c$.
It follows that $x\not\in V_X(I)$ and $I\not\subset \ker(x)$.
So we get a $g\in I$ with $g\not\in \ker(x)$.
We get $x\in D_X(g)$ and by corollary .. $D_X(g)\subset U$.
As stated earlier, $D_X(ab) = D_X(a) \cap D_X(b)$ since $\ker(x)$ is a prime ideal.
So $D_X(a)$ form a basis.
\end{proof}


\begin{comment}
\begin{lemma}
Let $U_i\subset X$ be an affine open of $X$.
Then in the sequence $I\tensor \sheaf{O}(U_i)\xrightarrow{j}\Lambda_X(I)(U_i) \xrightarrow{i} \sheaf{O}(U_i)$ the map $j$ is an iso and the map $i$ is injective
\end{lemma}
\begin{proof}
Let $I\xrightarrow{g}\sheaf{O}(X)$ be the inclusion.
The map $i$ is the component at $U_i$ of $\Lambda_X(i)$, which is injective by left exactness of $\Lambda_X$.
The inclusion $U_i\subset X$ is restrictive because $U_i$ is affine, hence $j$ is an isomorphism.
\end{proof}

\begin{remark}
I will use both $I\tensor \sheaf{O}(U_i)$ and $I \sheaf{O}(U_i)$ to denote the image (ideal) of the composition of these maps.
\end{remark}

\begin{lemma} Let $K,L\subset \sheaf{O}(X)$ be ideals. Then
\[V_X(K)\subset V_X(L) \Rightarrow \sqrt{L}\subset \sqrt{K}.\]
\end{lemma}
\begin{proof}
Assume \[V_X(K)\subset V_X(L).\]
Let $U_i = \Spec A_i \rightarrow X$ be a family of affine opens that covers $X$.
Note that $V_X$ restricts to the familiar 'zero-locus' function $V(\cdot)$ on any affine open.
Hence on $U_i$, we get
\[V(K A_i)  \subset V(L A_i).\]

For any ideal $N\subset \sheaf{O}(X)$, we get $\sqrt{N (A_i)_d}=\sqrt{N} (A_i)_d$ for every $d\in A_i$.
This implies $\Lambda_{U_i}(\sqrt{N} A_i)= \Lambda_{U_i}(\sqrt{N A_i})$ as follows.
Let $a\in \sqrt{N A_i}$, then we can get a finite cover of $U_i$ by distinguished opens which provides a matching family of restrictions $a_d\in \sqrt{N (A_i)_d}=\sqrt{N} (A_i)_d$.
By the sheaf property $a\in \sqrt{N} A_i$.
We use this for
\[\sqrt{L A_i} = \sqrt{L} A_i \subset \sqrt{K A_i} = \sqrt{K} A_i\] 
as ideals of $A_i$.
Hence we get an injection
\[\restr{\Lambda_X(\sqrt{L})}{U_i} \xrightarrow{f_i} \restr{\Lambda_X(\sqrt{K})}{U_i}\] 
that respects the inclusions \[g:\Lambda_X(\sqrt{L}) \rightarrow \sheaf{O}\]
and \[h:\Lambda_X(\sqrt{K}) \rightarrow \sheaf{O}.\] 
This ensures moreover that $f_i = f_j$ on $U_{ij}$ because $h$ is a mono.
We get an injection
\[\Lambda_X(\sqrt{L}) \xrightarrow{f} \Lambda_X(\sqrt{K})\]
by glueing. Since $\Lambda_X$ is an equivalence, we get an injection of modules $\sqrt{L} \rightarrow \sqrt{K}$ 
that respects the inclusions $\sqrt{K}\subset \sheaf{O}(X)$ and $\sqrt{L}\subset \sheaf{O}(X)$.
\end{proof}
\end{comment}
\begin{lemma}
The map $F$ is surjective.
\end{lemma}
\begin{proof}
Let $\Ip\in Y$ be a point in the target of $F$.
Then $\Lambda_X(\kappa(\Ip))$ is a quasi-coherent sheaf of modules.
In fact $\kappa(\Ip)\tensor_\Gamma(\sheaf{O}) \sheaf{O}(U)$ is a $\sheaf{O}(U)$ algebra, hence $\Lambda_X(\kappa(\Ip))$ is a quasi-coherent sheaf of algebras.
Hence we can compute the relative spec $\Rspec(\Lambda_X(\kappa(\Ip)))\rightarrow X$. The adjunct of the map
\[\Rspec(\Lambda_X(\kappa(\Ip)))\rightarrow Y\] is the canonical morphism $g:R\rightarrow \kappa(\Ip)$. 
This morphism is also the adjunct of the composition
\[\Rspec(\Lambda_X(\kappa(\Ip)))\rightarrow \Spec \kappa(\Ip)\rightarrow X,\]
so both maps must be equal. This gives us a commutative square

\begin{center}
\begin{tikzpicture}[node distance=3cm, auto]
  \node (C) {$X$};
  \node (B) [right of =C] {$Y$};
  \node (D) [below of= C]{$\Rspec(\Lambda_X(\kappa(\Ip)))$};
  \node (A) [below of= B]{$\Spec \kappa(\Ip)$};
  \draw[->] (C) to node {$F$} (B);
  \draw[->] (A) to node [swap] {}  (B);
  \draw[->] (D) to node [swap] {} (A);
   \draw[->] (D) to node [swap] {} (C);
\end{tikzpicture}
\end{center}

By lemma .., we know that $\Lambda_X(\kappa(\Ip))$ is not the zero sheaf hence the structure sheaf of $\Rspec(\Lambda_X(\kappa(\Ip)))$ non-zero.
This implies that the scheme is not the empty scheme. Therefore the point $\Ip$ is in the image of $F$.
\end{proof}

\begin{lemma}
The closed set $V_X(\Ip)$ is irreducible. This implies that $F$ is injective.
\end{lemma}
\begin{proof}
Let $F(z)=\Ip$ for some $z\in X$. By lemma .. this is possible.
Let $y\in V_X(\Ip)$. Then $\ker(z)\subset \ker(y)$, hence if $y\in D_X(a)$ then $x\in D_X(a)$. 
Therefore $y$ specialises to $z$, which thus must be $V_X(\Ip)$. This shows that it is irreducible.
Uniqueness of generic points of closed irreducible subsets of schemes implies injectivity of $F$.
\begin{comment}
Let $U \cup V = V_X(\Ip)$ with $U,V$ closed. 
The inclusions $u,v$ of closed sets $U,V$ is a quasi-compact map, because a closed set of a quasi-compact is quasi-compact. 
Therefore $u_*,v_*$ sends quasi-coherent modules to quasi-coherent modules [Reference].

Let $\sheaf{I},\sheaf{J}$ be the ideal sheaves corresponding to the reduced induced subscheme structure $(U,\sheaf{O}_U),(V,\sheaf{O}_V)$
on the closed sets $U$ and $V$. Hence the sequences
\[\sheaf{I}\rightarrow \sheaf{O}_X \rightarrow u_*(\sheaf{O}_U)\rightarrow 0,\]
\[\sheaf{J}\rightarrow \sheaf{O}_X \rightarrow v_*(\sheaf{O}_V)\rightarrow 0\]
are exact and so $\sheaf{I},\sheaf{J}$ are quasi-coherent.
Let $I,J$ be the global sections of $\sheaf{I}$ and $\sheaf{J}$.
Because of our adjoint equivalence $\Lambda_X(I)=\sheaf{I}$ and $\Lambda_X(J)=\sheaf{J}$, hence also $U=V_X(I)$ and $V=V_X(J)$.
We now have $V_X(IJ)=V_X(\Ip)$.
Moreover $\Lambda_X(\frac{\sheaf{O}(X)}{I}) \iso \frac{\sheaf{O}}{\sheaf{I}} \iso u_*(\sheaf{O}_U)$ and similar for $J,v,V$.
Since $\sheaf{O}_U$ is a reduced sheaf, $\sheaf{I}(W)$ is radical for every open $W \subset X$.

Now invoke the previous lemma for both inclusions wrt $IJ$ and $\Ip$. 
This yields $I\cap J = \Ip$, hence $I=\Ip$ or $J=\Ip$.
Now one of the closed sets $U,V$ is the whole of $V_X(\Ip)$.
\end{comment}
\end{proof}

\begin{comment}
\begin{definition}
Define $G(\Ip)$ to be the generic point of $V_X(\Ip)$.
\end{definition}
\begin{lemma}
The functions $F$ is a bijection.
\end{lemma}
\begin{proof}
Let $F(z)=\Ip$.
Note that $a\in F(G(\Ip))$ \iff $a\in \ker(x)$ for all $x\in V_X(\Ip)$.
We have
\begin{align*}
a\in \Ip &\Rightarrow a\in \ker(x), \forall x\in V_X(\Ip)\\
&\Rightarrow D_X(a) \cap V_X(\Ip) =\emptyset\\
&\Rightarrow a\in \ker(z)=\Ip\\
\end{align*}

Hence $FG = id_X$ and $F$ is surjective so a bijection.


Let $x\in X$. By construction $x\in V_X(F(x))$, because $F(x)=\ker(x)$.
Let $w\in V_X(F(x))$. Then $\ker(\sheaf{O}(X)\rightarrow \kappa(x))\subset \ker(\sheaf{O}(X)\rightarrow \kappa(w))$, so any open nhood $D_X(a)$ of $w$ contains $x$.
Therefore $x$ is the generic point of $V_X(F(x))$. Hence $GF = id_X$.


\end{proof}
\end{comment}

\begin{lemma}
The function $F$ is open, hence a homeomorphism.
\end{lemma}
\begin{proof}
Note that $F(D_X(a))= \{F(x) \st a\not\in F(x)\} = F(X)\cap D_Y(a) = D_Y(a)$.
Our map $F$ is continuous and open, so a homeomorphism.
\end{proof}

\begin{lemma}
If $F$ is a homeomorphism, then $X$ is affine.
\end{lemma}
\begin{proof}
%If F homeo, then F iso of LRS.
Let $\Spec A_i = U_i\subset X$ be open and let $\union_i U_i = X$.
Assume it is a finite affine cover.
Using our base, we get a cover of $U_i = \union_j D_X(a_{ij})$ with $a_{ij}$ global sections.
Observe that $D_X(a_{ij})\subset U_i$, hence $D(\restr{a_{ij}}{U_i}) = D_X(a_{ij})$ which makes them affine.
Continuing like this, we get a finite cover of affines $D_X(a_{ij})$ of $X$.
since $F(X)=F(\union_{ij} D_X(a_{ij})) = \union_{ij} D_Y(a_{ij}) = Y$, we have $(a_{ij})=(1)$.
Affine-ness satisfies the two requirements for the affine communication lemma[HAG II Ex.2.17], 
hence $X$ is affine.
\end{proof}