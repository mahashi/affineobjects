\chapter{Introduction}

\section{Motivation}
Quasi-coherent sheaves, affine schemes and the inclusion of distinguised opens of an affine scheme have a special relationship. 
A sheaf of modules $\sheaf{F}$ on any scheme $X$ is quasi-coherent 
\iff for any open affine subset $\Spec{R}\subset X$ and $f\in R$ 
we have the following coherence property $\sect{D(f)}{F} \tensor_{\sect{\Spec{R}}{O}} \rightarrow \sect{\Spec{R}}{F}$.
One can find out that a quasi-compact scheme is affine by proving that all higher cohomology of any quasi-coherent sheaf vanishes \cite[\href{http://stacks.math.columbia.edu/tag/01XF}{Tag 01XF}]{stacks}.

\section{Aim}
This thesis will investigate, formalize and generalize this trinity. 
We will generalize affineness to a property that any object of any ringed site may have, which gives us a notion of `affine' object, which we will call an \textbf{caffine}, shorthand for categorically affine, object. 
This definition will use the functor $\stilde$ which is a generalized version of the well-known functor $\widetilde{\blank}$ that sends modules of the global sections to a quasi-coherent module.
We will define a property  of morphisms in a ringed site, to replace the distinguised inclusions in the generalized trinity.
Both definitions are defined purely in terms of quasi-coherent sheaves.

Our main results are the following.
Any morphism between caffine objects will turn out to be restrictive.
If a ringed site has `enough caffines', then a sheaf module is quasi-coherent \iff
it is locally of the form $\stilde{M}$ for some module $M$.

\section{Outline}
In chapter 2 we will build up all the necessary theory that is well-known, but not assumed to be familiar.
We will start with some general categorical results and definitions.
Then we will introduce the generalisation of topology for categories, the category of sites and the central notion of a sheaf. The induced topology on an over/slice category will get special attention because we will use it extensively.

In chapter 3 we will introduce restrictive morphisms and show how this notion interacts with affine schemes.
It will turn out any morphism between affine schemes is restrictive.

In Chapter 4 we will define caffine objects 
and prove that any morphism between caffine objects is restrictive.
Then we will show that the caffine objects in the category of schemes are exactly the affine schemes.
Lastly we prove that a sheaf of modules is quasi-coherent \iff it is locally of the form $\stilde{M}$. We will also look at some examples of categories that lack any caffine objects.

In Chapter 5 I will expand on possible avenues of further research.

\section{Assumptions}
The intended audience are my peers, so some mathematical maturity is assumed.
I assume familiar with the usual undergraduate curriculum in topology and algebra. Furthermore basic categorical notions such as limits and adjunctions are assumed.
See the first 3 chapters of \cite{catsUU} for an introduction to these ideas.