\chapter{Introduction}

\section{Motivation}
Quasi-coherent sheaves and affine schemes have a special relationship. One can find out that a sheaf of modules on any scheme is quasi-coherent by looking at the sections at the affine open subsets. One can find out that a quasi-compact scheme is affine by proving all higher cohomology of any quasi-coherent scheaf vanishes \cite[Tag 01XF]{stacks}. 

\section{Aim}
This thesis will investigate 
and formalize this special relationship in 
a definition that only depends on the category of quasi-coherent sheaves on a scheme. 
This generalizes affine schemes to any ringed site, which gives us a notion of `affine' object, which we will call \textbf{caffine}, shorthand for categorically affine.
We will investigate how fruitful this definition is in other contexts.

\section{Outline}
In chapter 2 we will build up all the necessary theory that is well-known, but not assumed.
We will start with some general categorical results and definitions. Then we will introduce the generalisation of topology for categories, the category of sites and the central notion of a sheaf. The induced topology on an over/slice category will get special attention because we will use it extensively.

In chapter 3 we will introduce restrictive morphisms and show how this notion interacts with affine schemes.
It will turn out any morphism between affine schemes is restrictive.

In Chapter 4 we will define caffine objects.
Then we will show that the caffine objects in the category of schemes are exactly the affine schemes.
Lastly we will look at some examples of categories that lack any caffine objects.

In Chapter 5 I will expand on possible avenues of further research.

\section{Assumptions}
The intended audience are my peers, so some mathematical maturity is assumed.
I assume familiar with the usual undergraduate curriculum in topology and algebra. Furthermore basic categorical notions such as limits and adjunctions are assumed.
See the first 3 chapters of \cite{catsUU}.