
\begin{definition}\label[definition]{3be}
An object $B$ in a category $\cat{C}$ is called a \emph{generator} if for any distinct parallel morphisms 
$f,g:A\rightrightarrows C$ there exists a morphism $h:B\rightarrow A$ such that $g\circ h\neq f\circ h$.
Equivalently in an abelian category, an object $B$ is called a generator if for any proper(?) subobject $N<M$ there exists a map $B\rightarrow M$ that does not factor through $N$.
\end{definition}

\begin{definition}[Grothendieck Category]\label[definition]{3c4}
An abelian category $\cat{C}$ is called a Grothendieck category if filtered colimits are exact and the category has a generator.
\end{definition}

\begin{lemma}\label[lemma]{3c8}
The abelian category $R-Mod$ is Grothendieck for any ring $R$.
\end{lemma}
\begin{proof}
%Ring is generator for module category
The module $R$ is a generator since any element $m\in M$ of any module $M$ is in the image of the morphism $1\mapsto m$.
For any submodule $N$ of any module $M$, the module $R$ admits a map whose image intersects the complement of this submodule.

Filtered colimits are exact: \url{http://stacks.math.columbia.edu/tag/00DB}
\end{proof}

\begin{corollary}[\url{http://stacks.math.columbia.edu/tag/077K}]\label[corollary]{3d3}

Let $X$ be 'affine'. Then $\Qcoh{X}$ is a Grothendieck abelian category.
\end{corollary}

\begin{remark}[\url{http://stacks.math.columbia.edu/tag/077K}]\label[remark]{3d8}

Due to a result of Gabber, $\Qcoh{X}$ is a Grothendieck category for any scheme $X$.
\end{remark}

\begin{remark}\label[remark]{3dd}
The opposite of a Grothendieck abelian category is abelian but not Grothendieck. 
It is abelian since dualizing all requirements to be abelian yields the same set of requirements. 
It is not Grothendieck if R is non-zero. [Abelian Categories Freyd p116].
\end{remark}

\begin{lemma}\label[lemma]{3e3}
Let $\cat{A}$ be a Grothendieck abelian category.
Let $\presheaf{F}$ be a presheaf on $\cat{A}$.
Then $\presheaf{F}$ is representable \iff $\presheaf{F}$ commutes with colimits.
\end{lemma}
\begin{proof}[\url{http://stacks.math.columbia.edu/tag/07D7}]
By definition of colimits all presentable functors commute with colimits. 
Assume $\presheaf{F}$ commutes with colimits.
% step 1: F is group-valued, additive and right exact


% step 2: Define A and s_{univ}. Het A' becausee A is grothendieck and 
\end{proof}

\begin{lemma}\label[lemma]{3f1}
Let $\sheaf{G}$ be any $\sheaf{O}_X$ module. 
The functor $Q_{\sheaf{G}}:\Opposite{\Qcoh{X}}\rightarrow \cat{Sets}$ defined by
\[\sheaf{F}\mapsto \Hom{\sheaf{F}}{\sheaf{G}},\]
commutes with colimits, hence presentable.
\end{lemma}
\begin{proof}
Let $D:\Opposite{I}\rightarrow \cat{\Qcoh{X}}$ be a functor. 
Consider $\Hom{\colim(D)}{\sheaf{G}}$. The inclusion functor $i:\Qcoh{X}\rightarrow \cat{Mod(X)}$ commutes with colimits. This means that taking colimits in $\Qcoh{X}$ first and then applying $i$ yields the same as first applying $i$ component wise and then taking the limit.
Consider our hom-set inside $\cat{Mod(X)}$. Then by definition the colimit turns into a limit where the components are all hom-set of quasi-coherent modules into $\sheaf{G}$. 
\end{proof}

\begin{definition}[Right Adjoint/coherator]\label[definition]{3fd}
Define the functor $Q:\cat{Mod(X)}\rightarrow \Qcoh(X)$ that associates to a module $\sheaf{G}$ the presenting object of $Q_{\sheaf{G}}$.
Concretely $Q= \Lambda_X \circ \Gamma(-)$.
\end{definition}

\begin{lemma}\label[lemma]{402}
The functor $Q$ is right adjoint to the inclusion functor $i:\Qcoh{X}\rightarrow \cat{Mod(X)}$.
\end{lemma}

\begin{lemma}[\url{http://stacks.math.columbia.edu/tag/015Z}]\label[lemma]{406}
The functor $i$ is exact, hence $Q$ sends injectives to injectives. 
\end{lemma}


\color{red}
Rediscovered the adjunction between $\Gamma$ and $\Lambda$. 
Also rediscovered that the global sections of an injective sheaf are an injective module.
\color{red}