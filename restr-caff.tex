%TODO
    %%%%%%%%%%%%
    % PACKAGES %
    %%%%%%%%%%%%

% book example for classicthesis.sty
\documentclass[
  % Replace twoside with oneside if you are printing your thesis on a single side
  % of the paper, or for viewing on screen.
  %oneside,
  oneside,
  11pt, a4paper,
  footinclude=true,
  headinclude=true,
  cleardoublepage=empty
]{scrbook}

\usepackage[english]{babel}
\setlength\parindent{0pt}
\setlength{\parskip}{2.5mm}
\usepackage[T1]{fontenc}
\usepackage[utf8]{inputenc}
% \usepackage{tikz}
\usepackage{tikz-cd}
\usepackage{csquotes}
\usepackage{biblatex}
\addbibresource{affineobjects.bib}
% \usetikzlibrary{external}
% \tikzexternalize
\usepackage{hyperref}
\usepackage{url}
\usepackage{verbatim}
\usepackage{amsmath, amsthm, amsfonts, mathrsfs,bbm}
\usepackage{amssymb}
\usepackage{import}
\usepackage{mathtools}
\usepackage{eulervm}
\usepackage{concrete}
\usepackage{wasysym}
\usepackage{xparse}
%\usepackage[linedheaders,parts,pdfspacing]{classicthesis}

    %%%%%%%%%%
    % MACROS %
    %%%%%%%%%%

\newcounter{dummy}
\theoremstyle{plain}% default
\newtheorem{theorem}{Theorem}
\newtheorem{lemma}[theorem]{Lemma}
\newtheorem{question}[theorem]{Question}
\newtheorem{proposition}[theorem]{Proposition}
\newtheorem{corollary}[theorem]{Corollary}
\theoremstyle{definition}
\newtheorem{definition}[theorem]{Definition}
\newtheorem{notation}[theorem]{Notation}
\newtheorem{conjecture}[theorem]{Conjecture}
\newtheorem{example}[theorem]{Example}
\newtheorem{nonexample}[theorem]{Non-Example}
\newtheorem{exercise}[dummy]{Exercise}
\newtheorem*{solution}{Solution}
\newtheorem{construction}[theorem]{Construction}
\theoremstyle{remark}
\newtheorem*{remark}{Remark}

\newtheorem*{todo}{\begin{color}{red}{TODO}\end{color}}
\newtheorem*{Qanswer}{\textbf{Answer}}
\newtheorem{note}{Note}
\newtheorem*{case}{Case}

% solution environment
\newenvironment{answer}[1]
{\section*{Exercise #1}}
{}

% sub solution environment
\newenvironment{subanswer}[1]
{\subsection*{#1}}
{\qed}




%mathbb
\newcommand{\F}{\ensuremath{\mathbb{F}}}
\newcommand{\N}{\ensuremath{\mathbb{N}}}
\newcommand{\Z}{\ensuremath{\mathbb{Z}}}
\newcommand{\Q}{\ensuremath{\mathbb{Q}}}
\newcommand{\R}{\ensuremath{\mathbb{R}}}
\newcommand{\C}{\ensuremath{\mathbb{C}}}
\newcommand{\A}{\ensuremath{\mathbb{A}}}
\renewcommand{\S}{\ensuremath{\mathbb{S}}}
\renewcommand{\P}{\ensuremath{\mathbb}{P}}
\newcommand{\G}{\ensuremath{\mathbb}{G}}

\newcommand{\indicator}{\ensuremath{\mathbbm}{1}}


% Mathbf
\newcommand{\bN}{\ensuremath{\mathbf{N}}}
\newcommand{\bZ}{\ensuremath{\mathbf{Z}}}
\newcommand{\bQ}{\ensuremath{\mathbf{Q}}}
\newcommand{\bF}{\ensuremath{\mathbf{F}}}
\newcommand{\bR}{\ensuremath{\mathbf{R}}}
\newcommand{\bC}{\ensuremath{\mathbf{C}}}
\newcommand{\bS}{\ensuremath{\mathbf{S}}}
\newcommand{\bP}{\ensuremath{\mathbf}{P}}


% calligraphic
\newcommand{\curly}[1]{\ensuremath{\mathcal{#1}}}
\newcommand{\topo}[1]{{\ensuremath{\mathcal{#1}}}}
\newcommand{\cat}[1]{{\ensuremath{\mathsf{#1}}}}
\newcommand{\curlyK}{\ensuremath{\mathcal{K}}}
\newcommand{\curlyH}{\ensuremath{\mathcal{H}}}
\newcommand{\curlyF}{\ensuremath{\mathcal{F}}}
\newcommand{\curlyT}{\ensuremath{\mathcal{T}}}
\renewcommand{\O}{\ensuremath{\mathcal{O}}}
\newcommand{\curlyR}{\ensuremath{\mathscr{R}}}
\newcommand{\curlyP}{\ensuremath{\mathscr{P}}}
\newcommand{\curlyA}{\ensuremath{\mathscr{A}}}
\newcommand{\curlyS}{\ensuremath{\mathscr{S}}}
\newcommand{\curlyZ}{\ensuremath{\mathscr{Z}}}
\newcommand{\curlyI}{\ensuremath{\mathscr{I}}}



% frakd
\newcommand{\sheaf}[1]{\ensuremath{\mathfrak{#1}}}
\newcommand{\presheaf}[1]{\ensuremath{\mathfrak{#1}}}
\newcommand\psh\presheaf
\newcommand{\Ip}{\ensuremath{\mathfrak{p}}}
\newcommand{\Iq}{\ensuremath{\mathfrak{q}}}
\newcommand{\Ia}{\ensuremath{\mathfrak{a}}}
\newcommand{\Ib}{\ensuremath{\mathfrak{b}}}
\renewcommand{\Im}{\ensuremath{\mathfrak{m}}}
\newcommand{\shf}{\ensuremath{\mathfrak{F}}}
\newcommand{\Fs}{\ensuremath{\mathfrak{F}}}
\newcommand{\Is}{\ensuremath{\mathfrak{I}}}
\newcommand{\Js}{\ensuremath{\mathfrak{J}}}
\newcommand{\shg}{\ensuremath{\mathfrak{G}}}
\newcommand{\Gs}{\ensuremath{\mathfrak{G}}}
\newcommand{\Nil}{\ensuremath{\mathfrak{N}}}


% Delimiters
\DeclarePairedDelimiter\abs{\lvert}{\rvert}
\DeclarePairedDelimiter\brac[]
\DeclarePairedDelimiter\cbrace\{\}
\DeclarePairedDelimiter\paren()
\DeclarePairedDelimiter{\gen}\langle\rangle
\DeclarePairedDelimiter{\nrm}\lVert\rVert
\DeclarePairedDelimiter{\card}\lVert\rVert

% custom commands
%\newcommand{\colim}[1]{\underset{#1}{colim}}
\newcommand{\limit}[1]{\underset{#1}{lim}}
\newcommand{\se}{\ensuremath{\subseteq}}
\newcommand{\tbs}{\textbackslash }
\newcommand{\sh}{\backslash}
\newcommand\restr[2]{{% we make the whole thing an ordinary symbol
  \left.\kern-\nulldelimiterspace % automatically resize the bar with \right
  #1 % the function
  \vphantom{\big|} % pretend it's a little taller at normal size
  \right|_{#2} % this is the delimiter
  }}

\newcommand\compl[1]{{#1}^c}
\newcommand\restrict{\restr}
\newcommand{\iso}{\cong}
\newcommand{\kr}{\mbox{Ker}}
\newcommand{\im}{\mbox{Im}}
\newcommand{\Sup}{\mbox{Supp}}
\newcommand{\Ht}{\mathfrak{h}\mathfrak{t}}
\newcommand{\Dim}[1]{\mbox{Dim}(#1)}
\newcommand{\Span}[1]{\mbox{Span}_{#1}}
\renewcommand{\deg}{\mbox{deg }}
\newcommand{\eq}{\sim}
\renewcommand{\to}{\mapsto}
\newcommand{\givenby}{,\quad}
%\newcommand{\Spec}[1]{\mbox{Spec }{#1}}
\newcommand{\Rspec}{\mbox{Rspec }}
\newcommand{\Ann}{\mbox{Ann}}
\newcommand{\Der}{\mbox{Der}}
\newcommand{\Trdg}{\mbox{Tr.dg }}
\newcommand{\Proj}{\mbox{Proj }}
\newcommand{\Sp}{\mbox{sp }}
\newcommand{\Hom}[2]{\mbox{Hom}(#1,#2)}
%\newcommand{\Hom}[3]{\mbox{Hom}_{#1}(#2,#3)}
\newcommand{\Dom}[1]{\mbox{Dom}(#1)}
\newcommand{\coproduct}{\bigoplus}
\newcommand{\product}{\prod}
\newcommand{\presheafcat}[1]{\widehat{#1}}

\newcommand{\Dist}{\mbox{\textbf{Dist}}}
%Category theory

\newcommand{\Sch}{\cat{Sch}}
\newcommand\Schemes\Sch
\newcommand{\Rng}{\cat{Rng}}
\newcommand\Rings\Rng
\newcommand{\CmRng}{\cat{CmRng}}
\newcommand\LocRingSpaces{\cat{LRSpaces}}

\newcommand{\union}{\bigcup}
\newcommand{\opUnion}{\cup}
\newcommand{\match}[2]{\mbox{Match}(#1,#2)}
\newcommand\matchfam[3]{\set{{#1}_{#2} \suchthat #2 \in #3}}
\newcommand{\cov}{\mbox{Cov}}
\newcommand{\Opp}[1]{{#1}^{Opp}}
\newcommand{\Opposite}[1]{{#1}^{Opp}}

% Differential operators
\renewcommand\qedsymbol{\rule{1ex}{1ex}}
\newcommand{\from}{\ensuremath{\colon}}
\renewcommand{\iff}{if and only if }
\newcommand{\tensor}{\otimes}
\newcommand{\heq}{\simeq}
\newcommand{\st}{\;|\;}


\newcommand{\copresheaves}[1]{\hat{\cat{#1}}}
\newcommand{\presheaves}{\copresheaves}
\newcommand{\functors}[2]{[\cat{#1},\cat{#2}]}
\newcommand{\opposite}[1]{\cat{#1}^{op}}
\newcommand{\opp}{\opposite}
%\newcommand{\coyoneda}[1]{h_{#1}}
\newcommand\ynd\yoneda
\newcommand\y\ynd
\newcommand{\contrayoneda}[1]{h^{#1}}
\newcommand{\overcat}[2]{\cat{#1}_{#2}}
\newcommand{\undercat}[2]{\cat{#1}^{#2}}
\newcommand{\oversite}{\overcat}
\newcommand{\undersite}{\undercat}
\newcommand{\obj}[1]{Obj(\cat{#1})}
\newcommand{\signature}[3]{#1:#2\rightarrow #3}
\newcommand{\suchthat}{ \mid }
\newcommand{\definedAs}{:=}
\newcommand{\basis}{\topo}
\newcommand{\intersect}{\cap}

\newcommand{\Sets}{\cat{Set}}
\newcommand{\opens}[1]{\cat{Open}(#1)}
\newcommand{\site}[2]{(\cat{#1}, \topo{#2})}
\newcommand{\ringedspace}[2]{({#1}, {\sheaf{#2}})}
\newcommand\rspace\ringedspace
\newcommand\scheme\ringedspace
\newcommand{\rsite}[3]{(\cat{#1}, \topo{#2}, {\sheaf{#3}})}


% \newcommand{\rsite}[2]{({\cat{#1}}, {\sheaf{#2}})}

\newcommand{\adjunction}[2]{{#1} \dashv {#2}}
\newcommand{\adj}{\adjunction}


%\newcommand{\stilde}[1]{\Lambda(#1)}
\DeclareDocumentCommand{\stilde}{ g g }{%
 \IfNoValueTF{#2}%
          {\IfNoValueTF{#1}%
            {\Lambda}%
            {\Lambda(#1)}%
          }%
          {\Lambda_{#1}(#2)}%
}
\DeclareDocumentCommand{\pstilde}{ g g }{%
 \IfNoValueTF{#2}%
          {\IfNoValueTF{#1}%
            {\lambda}%
            {\lambda(#1)}%
          }%
          {\lambda_{#1}(#2)}%
}

\DeclareDocumentCommand{\restrictOpt}{ m g }{%
 \IfNoValueTF{#2}%
          {\restrict{\pt}{#1}}%
          {\restrict{#2}{#1}}%
}

\DeclareDocumentCommand{\colimit}{ m g }{%
 \underset{#1}{\mbox{colim} } 
 \IfNoValueTF{#2} 
    {}
    {\; #2}%
}
\newcommand\colim\colimit
\DeclareDocumentCommand{\limit}{ m g }{%
 \underset{#1}{\mbox{lim} } 
 \IfNoValueTF{#2} 
    {}
    {\; #2}%
}

\DeclareDocumentCommand{\yoneda}{ g }{%
 \IfNoValueTF{#1}%
          {h}%
          {h(#1)}%
}

\DeclareDocumentCommand{\mainadjunction}{ g }{%
 \IfNoValueTF{#1}%
          {\adj{\stilde{\blank}}{\globalsections{\blank}}}%
          {\adj{\stilde{#1}{\blank}}{\sections{#1}}}%
}
%\newcommand{\mainadjunction}{\adj{\stilde{\blank}}{\globalsections{\blank}}}

\DeclareDocumentCommand{\sheafify}{ g }{%
 \IfNoValueTF{#1}%
          {sh}%
          {sh(#1)}%
}

\DeclareDocumentCommand{\Spec}{ g }{%
 \IfNoValueTF{#1}%
          {\mbox{Spec}}%
          {\mbox{Spec}(#1)}%
}
\newcommand\spec\Spec

\DeclareDocumentCommand{\plus}{ g }{%
 \IfNoValueTF{#1}%
          {(\blank)^+}%
          {{#1}^+}%
}

\DeclareDocumentCommand{\globalsections}{ g }{%
 \IfNoValueTF{#1}%
          {\sections{\terminal}{\blank}}%
          {\sections{\terminal}{#1}}%
}
\newcommand\gblsect\globalsections

\DeclareDocumentCommand{\sections}{ mg }{%
 \IfNoValueTF{#2}%
          {\Gamma( #1 ; \blank)}%
          {\Gamma( #1 ; \presheaf{#2})}%
}

\newcommand\sect\sections

\DeclareDocumentCommand{\top}{ m g }{%
 \IfNoValueTF{#2}%
          {\topo{#1}}%
          {\topo{#1}(#2)}%
}
\DeclareDocumentCommand{\sieve}{ m g }{%
 \IfNoValueTF{#2}%
          {#1}%
          {{#1}(#2)}%
}
\DeclareDocumentCommand{\incsh}{ g }{%
 \IfNoValueTF{#1}%
          {i}%
          {i(#1)}%
}


\newcommand\Supp{\mbox{Supp }}
\newcommand{\maxsieve}[1]{\cat{max}(#1)}
\newcommand{\sheaves}[1]{\cat{Shv}(#1)}
\newcommand{\sheafmodules}[1]{\cat{Mod}(#1)}
\newcommand\shmodules{\sheafmodules}
\newcommand{\pshmodules}[1]{\cat{PMod}(#1)}
\newcommand{\qcoh}[1]{\cat{Qcoh}(\sheaf{#1})}
%\newcommand{\plus}[1]{{\presheaf{#1}}^+}
%\newcommand{\sheafify}[1]{a(\presheaf{#1})}
\newcommand{\id}{\mbox{Id}}
\newcommand{\blank}{{-}}
\newcommand{\prshfs}[2]{\presheaves{#1}(\cat{#2})}
\newcommand\pshs\presheaves
\newcommand\shs\sheaves
%\newcommand{\sections}[2]{\Gamma( #1 ; \presheaf{#2} )}
%\newcommand{\globalsections}[1]{\sections{\terminal}{#1}}
\newcommand\spull[1]{{#1}^*}
\newcommand{\cover}[2]{\set{{#1} \rightarrow {#2}}}

\newcommand{\module}[1]{\sheaf{#1}}
\newcommand{\terminal}{1}
\newcommand{\initial}{0}
\newcommand\sievecat[1]{\cat{Sieves}({#1})}
\newcommand{\schemes}{\cat{Sch}}
\newcommand{\affschemes}{\cat{AffSch}}
\newcommand\sites{\cat{Sites}}
\newcommand\inverse[1]{{#1}^{-1}}
\newcommand{\direct}[1]{{#1}_*}
\newcommand{\inverseimg}[1]{{#1}^*}
\newcommand{\forward}[1]{{#1}_!}
\newcommand\pushforward\forward
\newcommand{\interhom}[2]{{#1}^{#2}}
\newcommand{\modules}[1]{#1\text{-}\cat{Mod}}
\newcommand{\pullback}[1]{{#1}^*}
\newcommand{\unit}{\eta}
\newcommand{\counit}{\epsilon}
\newcommand\subobjects[1]{\mbox{Sub}({#1})}
\newcommand{\domain}[1]{\mbox{Dom}(#1)}
\newcommand\dom{\domain}
\newcommand{\codomain}[1]{\mbox{Codom}(#1)}
\newcommand{\kernel}[1]{\mbox{Ker}(#1)}
\newcommand{\cokernel}[1]{\mbox{Coker}(#1)}
\newcommand{\image}[1]{\mbox{Im}(#1)}
\newcommand{\coimage}[1]{\mbox{Coim}(#1)}

\newcommand{\affspace}[1]{\A^{1}}
\newcommand{\projspace}[1]{\P^{1}}

\newcommand{\pt}{\ast}
\newcommand\eqclass[1]{[#1]}

\newcommand\set[1]{\{#1\}}
%\renewcommand{\if}{\mbox{ if }}
\newcommand{\otherwise}{\mbox{ otherwise }}
\newcommand{\reason}[1]{\mbox{ by #1}}
\newcommand\onsome{\mbox{ on some }}
\newcommand\with{\mbox{ : }}
\renewcommand\and{\mbox{,  }}
%------------------------------------------------------------------------------ 
% ADMINISTRATION 
%------------------------------------------------------------------------------ 
\newcommand{\homeworkset}{5} %%% UPDATE THIS NUMBER 
\title{Affine Objects}
\newcommand{\firstName}  {Mohamed} 
\newcommand{\lastName}   {Hashi} 
\newcommand{\studId}     {1408593} 
\newcommand{\email}    {maxamedhashi@gmail.com} 
\newcommand{\uni} {Universiteit Leiden}
\author{Mohamed Hashi}

\begin{document}

\section*{Prelims}

%declarations
Let $Y = \rsite{X}{T}{O}$ be a ringed site.
Let $R = \globalsections{O}$. Let $a,b,c,c'\in X$.
Let $\module{F}$ be a quasi-coherent module on $\oversite{Y}{a}$.
Let $M = \sections{a}{F} = \globalsections{F}$.
Let $f:b \rightarrow a$.

Some basic definitions and constructions.

\begin{definition}[Over/Under categories]
Let $\cat{C}$ and $\cat{C'}$ be categories. Let $F:\cat{C}\rightarrow \cat{C'}$ and $Z\in C'$. 
Define the category $\overcat{C}{Z}$ and $\undercat{C}{Z}$ as follows
\[\obj{\overcat{C}{Z}} := \{(a,w) \suchthat a\in \cat{C}, w:F(a)\rightarrow Z\},\]
\[\Hom{(a,w)}{(Y,v)} := \{f:a\rightarrow Y \suchthat v\circ F(f) = w  \},\]

and

\[\obj{\undercat{C}{Z}} := \{(a,w) \suchthat a\in \cat{C}, w:Z\rightarrow F(a)\},\]
\[\Hom{(a,w)}{(Y,v)} := \{f:a\rightarrow Y \suchthat F(f)w = v  \}.\]

We get faithfull functors 
$\overcat{C}{Z} \rightarrow \cat{C}: (a,w)\rightarrow a$ and 
$\undercat{C}{Z} \rightarrow \cat{C}: (a,w)\rightarrow a$.
We will call both functors $u$ and suppress the functor $F$ where there can be no confusion.
\end{definition}


\begin{definition}
Let $M,N$ be an $R$-module.
Let $g:M \rightarrow N$.
Define 
\[\pstilde: \modules{R} \rightarrow \pshmodules{Y}\]
by
\[\pstilde(M)(a) = M \tensor_R \sections{a}{O} ,\]
\[\pstilde(M)(f): \id \tensor \sheaf{O}(f) ,\]
\[\pstilde(g) = (a: g \tensor \id ).\]
\end{definition}

\begin{definition}
Define
\[\stilde: \modules{R} \rightarrow \shmodules{Y}\]
by
$\sheafify \circ \pstilde$.
\end{definition}

This functor is left adjoing to the global sections functor, which I will prove in the next episode.



\begin{lemma}
Let $X$ be a category.
Let $f\in X$.
Let $S$ be a collection of sets indexed by the objects of $X$
and $S(b)$ be a subset of $\Hom{b}{f}$.
Then $u(S)$ is a sieve on $u(f)$ \iff $S$ is a sieve on $f$.
\end{lemma}

\begin{proof}
=>:
Let $h: d \rightarrow b \in S$ 
and $k: e \rightarrow d$ be arbitrary.
By assumption $u(hk)\in u(S)$.
The functor $u$ is faithfull, so $hk \in S$.

<=:
Let $h: d \rightarrow u(f) \in u(S)$ 
and $k: e \rightarrow d$ be arbitrary.
By assumption $hk \in S$, hence $u(hk) \in u(S)$.
\end{proof}

We will define the induced topology $\top{S}$ on $\overcat{C}{a}$.
That $u$ considered as a map on sieves commutes with the pullback of sieves is used and will not be proved.

\begin{definition}
Let $\top{T}{u(f)}$ be the set of covering sieves on $u(f) \in a$.
By the previous lemma sieves on $u(f)$ are sieves on $f$.
Let $\top{S}{f} = \{R \suchthat u(R) \in \top{T}{u(f)}$ be the induced topology.
So $u(R)$ is covering on $u(f)$ \iff $R$ is covering on $f$.

a) Since u commutes with pullback of sieves, 
we have $\maxsieve{u(f)} = u(\maxsieve{f}) = \maxsieve{f}$,
hence $\maxsieve{f} \in \top{S}{f}$.

b) Let $R$ be a covering sieve on $f$.
Let $h:b'\rightarrow a$
and $p:b' \rightarrow b$ with $fp = h$.
Commutativity of $u$ and pulling back implies that $u(p)^*u(R) = u(p^*R)$.
Hence $p^*R$ is covering since $u(p^*R)$ is.

c) Let $R$ be a covering sieve on $f$
and $Q$ be a sieve on $f$.
Let $h:b'\rightarrow a$
and $p:b' \rightarrow b \in R$,
hence with $fp = h$.
Assume $p^*Q$ is covering for every such $p$.
Then $u(p^*Q) = u(p)^*u(Q)$ is covering for every $p$.
We know that $u(R)$ is covering hence $u(Q)$ must be, which implies that $Q$ is covering.

We proved that $\top{S}$ is indeed a Grothendieck topology.
\end{definition}

\section*{Main}

\begin{lemma}
Let $a$ be caffine.
The global component of the sheafification morphism is equal to the unit of $\adj{\stilde}{\globalsections}$ in $\overcat{C}{a}$.
\end{lemma}

\begin{proof}
Let $M$ be a $\sections{a}{O}$-module.
Consider the following maps, which you get by repeatedly calling on an adjunction bijection.
Let $i$ be the universal sheafification morphism.

\[\stilde(M) \rightarrow \stilde(M)\]
\[	i : \pstilde(M) \rightarrow \stilde(M) \mbox{ use sheafification adjunction}\]
\[ M \rightarrow \sections{a}{\stilde(M)} \mbox{ use } \adj{\pstilde}{\sections{a}}\]

If you compose the two adjunction bijections used, 
you get the bijection of $\adj{\stilde}{\sections{a}}$
by definition, so the last map is actually $\unit_M$.
Hence $i_a = \unit_{M}$, which is an iso by assumption.
\end{proof}

\begin{proposition}
The adjunct of $f$
\[\sections{a}{F} \tensor_{\sections{a}{O}} \sections{b}{O} \rightarrow \sections{b}{F}\]
is an isomorphism.
\end{proposition}
Consider
\begin{center}
 	\begin{tikzcd}[row sep = large, column sep = large]
		\pshmodules{\oversite{Y}{a}} \arrow{r}{\sheafify_a}\arrow{d}{\restrictOpt{b}} 
		& \shmodules{\oversite{Y}{a}} \arrow{d}{\restrictOpt{b}}
		& \modules{\sections{a}{O}} \arrow{l}{\stilde} \arrow[bend right]{ll}{\pstilde} 
		\arrow{d}{\blank \tensor \sections{b}{O}}\\
		\pshmodules{\oversite{Y}{b}} \arrow{r}{\sheafify_a}
		& \shmodules{\oversite{Y}{b}}
		& \modules{\sections{b}{O}} \arrow{l}{\stilde}\arrow[bend left]{ll}{\pstilde}

 	\end{tikzcd}
 \end{center}

By a previous lemma, the left square commutes. 
By definition the two `triangles' commute too and the outer square commute,
hence the right square also commutes. 
Therefore $M \tensor \sections{b}{\sheaf{O}} \iso \sections{b}{\sheaf{F}}$.
This is the proof you wrote down friday.

The requirement is not to find any isomorphism but a specific one.
So I think this is not enough and we need to do some bookkeeping and see if the witnessing isomorphism is our map.

Consider
 \begin{center}
 	\begin{tikzcd}[row sep = large, column sep = large]
 		\restrictOpt{b}{\stilde{M}}
 		& \stilde( M \tensor \sections{b}{O}) \arrow{l}{s} \\
 		\pstilde( M \tensor \sections{b}{O}) \arrow{u}{\restrictOpt{b}{i}} \arrow{ru}{j}
 	\end{tikzcd}
 \end{center}

The natural transformation $j$ is the universal sheafification morphism coming from $\sheafify_b$.
We have seen that $\sections{b}{j}$ and $s$ are isomorphisms

Let $g: c \rightarrow b$.
Let $x = m \tensor r \in \pstilde( M \tensor \sections{c}{O})$.
Then $j_g(x) = (x_{i})$ indexed by the maximal sieve on $g$
and $i_g(x) = i_{fg}(x) = (x_{i})$ indexed by the maximal sieve on $gf$.
Hence we get  $s_g(j_g(x)) = i_g(x)$, so the triangle commutes.
Evaluating everything on the terminal, in this case on $b$, 
shows that two out of three maps are isomorphisms, hence $i_b$ is an isomorphism.

\end{document}