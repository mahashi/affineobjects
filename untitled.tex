\begin{document}


\begin{lemma}
Let $C$ be a finite poset. Every element of $C$ has a affine cover.
\end{lemma}

%Cover = Covering sieve
Let $x_0\in C$. 
If $x_0$ is covered by the maximal sieve only or the maximal sieve and the empty sieve, it is affine and we are done. 
Assume otherwise.
Let $S = \{y_i \rightarrow x_0\}$ be a non-trivial cover of $x_0$.
Then $S$ does not contain any isomorphisms $y\rightarrow x_0$.
Let $S_i\{y_{ij}\rightarrow y_i\}$ be a non-trivial cover of non-affine $y_i$.

We can associate to any non-maximal non-empty covering sieve $S$ of an element $x_0$,
the set of all NA-chains $x_0\leftarrow x_1 \leftarrow \ldots \leftarrow x_n$.
An NA-chain, associated to $S$, is a chain of maps ending in $x_0$ such that $x_i \leftarrow x_{i+1}$ is contained in a non-maximal, 
non-empty cover of $x_i$, where $x_0\leftarrow x_1$ is contained in $S$.

By finiteness of $C$, any chain of maps is bounded by the size of $C$ or contains a cycle. 
If a chain contains a cycle, it contains isomorphisms. 
By construction, no isos can be present in a NA-chain, since isos generate the maximal sieve. 
Therefore the length of any NA-chain is bounded by $\card{C}$.

Let $H$ be a chain of maximal length $m$. 
Then the last map $h \leftarrow g$ in $H$ has an affine object $g$ as domain.
Apparently $H$ cannot be appended, hence $g$ has no non-maximal, non-empty coverings hence is affine.
Also the non-maximal, non-empty covering of $h$ where this map appears must have be an affine covering by applying the same reasoning to the other objects occuring in it. 
Hence all objects occurring at the $m$th place in a chain admits a affine cover.
Let $i\leq m$ and $b$ be a object occuring at the $i-1$th place in a chain. 
It is either affine or all objects in any non-maximal, non-empty cover occur at the $i$th place hence admit an affine cover.
Therefore $b$ admits an affine cover. Hence $x_0$ admits a affine cover by reversed induction.




%Contradiction
Let $x$ be an element without a affine cover.
Then every cover $C$ contains a non-affine object.
Let $C_0 = \{y_i\rightarrow x\}$ a covering sieve of $x$.
Let $j$ be such that $y_j$ is non-affine.

Let $C_i = \{y_{ij} \rightarrow y_i\}$ be a cover of $y_i$ for all $i$.
If $y_i$ is not affine, let $C_i$ be non-empty and non-maximal.
Then $C = \{y_{ij} \rightarrow x\}$ is a cover of $x$, by the 'consistency'-axiom of a GR-topology.


\begin{counterexample}
The category is $\Z \times \Z$ with the usual ordering.
An element $(i,j)$ is only non-trivially covered by $\{(i,j-1)\rightarrow (i,j),(i-1,j)\rightarrow (i,j)\}$.
Let $R = k[x_{ij}| i,j\in \Z]$. Define the structure sheaf as $O(i,j) = R[x_{kl}^{-1}| k\leq i & l \leq j]$.

%this is a sheaf:


Define the sheaf of modules $F(i,j)= R[x_{kl}^{-1}| k< i & l < j]$. 

%this is a sheaf:

Restricted to (i,j), $F$ is quasi-coherent but not globally presentable, hence all objects are non-affine.

Consider $G = O(i,j)[y_{kl}| k\leq i & l \leq j]$. For any object $(i,j)$, the section $y_{i-1,j-1}$ of $G(i-1,j-1)$ is not generated by sections in  module $G(i-1,j-1)$. Hence the module $G$ is not generated by global sections.
\end{counterexample}
\end{document}